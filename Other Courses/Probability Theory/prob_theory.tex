\documentclass[11pt]{article}
\usepackage{spikey}
\usepackage{amsmath}
\usepackage{amssymb}
\usepackage{soul}
\usepackage{float}
\usepackage{graphicx}
\usepackage{hyperref}
\usepackage{xcolor}
\usepackage{chngcntr}
\usepackage{centernot}
\usepackage[shortlabels]{enumitem}

\usepackage[margin=1truein]{geometry}

\title{Notes on Probability Theory \\ 18.175}
\date{\today}
\author{Tianyu Du}

\counterwithin{equation}{section}
\counterwithin{theorem}{section}
\counterwithin{lemma}{section}
\counterwithin{corollary}{section}
\counterwithin{proposition}{section}
\counterwithin{remark}{section}
\counterwithin{example}{section}

\begin{document}
	\maketitle
	\tableofcontents
	
	\section{Preliminaries}
		\begin{definition}
			A \textbf{probability space} is a triple $(\Omega, \mc{F}, P)$ where $\Omega$ is the \textbf{sample space}, $\mc{F}$ is a $\sigma$-algebra of $\Omega$ (\textbf{events}) and $P: \mc{F} \to [0,1]$ is the \textbf{probability function}.
		\end{definition}

		\begin{remark}
			$(\Omega, \mc{F})$ is a \textbf{measurable space} or \textbf{Borel space}.
		\end{remark}
		
		\begin{definition}
			A \textbf{algebra}, $\mc{A}$, of set $X$ is a collection of subsets of $X$ closed under complementation and \emph{finite} union.
		\end{definition}
		
		\begin{definition}
			A \textbf{$\sigma$-algebra} of set $X$ is a collection of subsets of $X$ closed under complementation and \emph{countable} union.
		\end{definition}
		
		\begin{remark}
			We can also define \emph{algebra} and \emph{$\sigma$-algebra} using closures under complementation and \emph{finite/countable intersection}.
		\end{remark}
		\begin{proof}
			Use DeMorgan's Law.
		\end{proof}
		
		\begin{definition}
			A measure $\mu$ on $\mc{A}$ is \textbf{$\sigma$-finite} if there exists \emph{countable} collection $A_n \in \mc{A}$ with $\mu(A_n) < \infty$ and $\cup A_n = \Omega$.
		\end{definition}
		
		\begin{definition}
			A \textbf{semi-algebra} $\mc{S}$ is a collection of sets closed under intersection such that $S \in \mc{S}$ implies that $S^c$ is a \emph{finite disjoint} union of sets in $\mc{S}$.
		\end{definition}
		
		\begin{lemma}
			Let $\mc{S}$ be a semi-algebra, then
			\begin{equation}
				\overline{\mc{S}} = \tx{all \emph{finite disjoint unions} of sets in }\mc{S}
			\end{equation} is an algebra, called the \textbf{algebra generated by $\mc{S}$}.
		\end{lemma}
		\begin{proof}
			We are going to show the equivalent definition of algebra, that's, $\bar{\mc{S}}$ is closed under complementation and finite intersection. \\
			\emph{Intersection}: Let $A, B \in \bar{\mc{S}}$, then by definition of $\bar{\mc{S}}$,
			\begin{gather}
				A = \cup_i A_i\ A_i \in \mc{S} \\
				B = \cup_j B_j\ B_j \in \mc{S}
			\end{gather}
			Then by definition of semi-algebra, $A_i \cap B_j \in \mc{S}$. Then 
			\begin{gather}
				A \cap B = (\cup_i A_i) \cap (\cup_j B_j) \\
				= \cup_{i,j} A_i \cap B_j \in \bar{\mc{S}}
			\end{gather}
			By an inductive argument, we've shown that $\bar{\mc{S}}$ is closed under intersection. \\
			\emph{Complementation}: Let $A \in \bar{\mc{S}}$, by definition
			\begin{gather}
				A = \cup_i A_i\ A_i \in \mc{S}
			\end{gather}
			Therefore, by DeMorgan's Law, $A^c = \cap_i A_i^c$ and by definition of semi-algebra, for each $A_i^c$, it's a finite union of disjoint sets in $\mc{S}$. \\
			By definition of $\bar{\mc{S}}$, each $A_i^c \in \bar{\mc{S}}$. And as shown above, $\bar{\mc{S}}$ is closed under finite intersection.\\
			Therefore $A^c \in \bar{\mc{S}}$.\\
			So $\bar{\mc{S}}$ is closed under complementation. \\
			Therefore $\bar{\mc{S}}$ is an algebra.	
		\end{proof}
		
		\begin{definition}
			A \textbf{measure} on algebra is a function $\mu: \mc{A} \to \R$ such that
			\begin{enumerate}[(i)]
				\item $\mu(A) \geq \mu(\emptyset) = 0\ \forall A \in \mc{A}$,
				\item and countably additive for \emph{disjoint} set $\{A_i\}_i$
				\begin{equation}
					\mu(\cup_i A_i) = \sum_i \mu(A_i)
				\end{equation}
			\end{enumerate}
		\end{definition}
		
%		\begin{definition}
%			A \textbf{measure} on $\mc{F}$ is a function $\mu:\mc{F} \to \R$ satisfying $\mu(A) \geq \mu(\emptyset) = 0$ for all $A \in \mc{F}$. And $\mu$ is \emph{countably} additive for \emph{disjoint} $\{A_i\}_i$. That's
%			\begin{equation}
%				\mu(\cup_i A_i) = \sum_i \mu(A_i)
%			\end{equation}
%		\end{definition}
		
		\begin{definition}
			A measure $\mu$ on $\mc{F}$ is a \textbf{probability measure} if $\mu(\Omega) = 1$.
		\end{definition}
		
		\begin{definition}
			The \textbf{Borel $\sigma$-algebra} $\mc{B}$ on a topological space is the smallest $\sigma$-algebra \emph{containing all open sets}.
		\end{definition}
		
		\begin{theorem}
			For each \emph{right continuous, non-decreasing} function $F$ such that $\lim_{x \to -\infty} F = 0$ and $\lim_{x \to \infty} F = 1$, there is an \emph{unique} measure defined on the Borel sets of $\R$ with 
			\begin{equation}
				P((a,b]) \equiv F(b) - F(a)
			\end{equation}
		\end{theorem}
		
		\begin{definition}
			A collection $\mc{P}$ of sets is a \textbf{$\pi$-system} is it's closed under intersection.
		\end{definition}
		
		\begin{definition}
			A collection $\mc{L}$ of subsets of $\Omega$ is a \textbf{$\lambda$-system}(Dynkin system) if
			\begin{enumerate}[(i)]
				\item $\Omega \in \mc{L}$.
				\item (\emph{Closed under set difference}) If $A, B \in \mc{L} \land A \subseteq B \implies B \backslash A \in \mc{L}$.
				\item (\emph{Contain set sequence limit}) If $A_n \in \mc{L}$ and $A_n \uparrow A$, then $A \in \mc{L}$.
			\end{enumerate}
		\end{definition}
		
		\begin{theorem}
			If $\mc{P}$ is a $\pi$-system and $\mc{L}$ is a $\lambda$-system containing $\mc{P}$, then $\sigma(\mc{P}) \subseteq \mc{L}$, where $\sigma(\mc{A})$ denotes the smallest $\sigma$-algebra containing $\mc{A}$.
		\end{theorem}
		
		\begin{theorem}[Caratheodory Extension Theorem]
			If $\mu$ is a a $\sigma$-finite measure on an algebra $\mc{A}$, then $\mu$ has a \emph{unique} extension to the $\sigma$-algebra generated by $\mc{A}$.
		\end{theorem}
	
	\section{Random Variables}
		\begin{definition}
			A \textbf{measurable space} is a tuple $(S, \Sigma)$ where $\Sigma$ is a $\sigma$-algebra on $S$.
		\end{definition}
		
		\begin{remark}
			The definition of \emph{measurable spaces} does not require a specific measure.
		\end{remark}
		
		\begin{definition}
			Let $(X, \Sigma)$ and $(Y, \Pi)$ be two measurable spaces, and function $f: X \to Y$ is a \textbf{measurable function} if 
			\[
				\forall \mc{E} \in \Pi,\ f^{-1}(\mc{E}) \in \Sigma
			\]
			Denoted as $f: (X, \Sigma) \to (Y, \Pi)$.
		\end{definition}
		
		\begin{definition}
			A \textbf{random variable} is a measurable function $X: (\Omega, \mc{F}) \to (\R, \mc{B})$. We say $X$ is $\mc{F}$-measurable.
		\end{definition}
		
		\begin{theorem}
			If $X^{-1}(A) \in \mc{F}$ for all $A \in \mc{A}$ and $\mc{A}$ generates $\mc{S}$, then $X$ is a measurable map from $(\Omega, \mc{F})$ to $(S, \mc{S})$.
		\end{theorem}
		
		\begin{definition}
			Let $F_X(x) \equiv P(X \leq x)$ be the \textbf{distribution function} for $X$. And write $f = f_X = F'_X$ for the \textbf{density function} of $X$. The distribution function must be
			\begin{enumerate}[(i)]
				\item Non-decreasing
				\item Right-continuous
				\item $\lim_{x\to \infty} F(x) = 1$
				\item $\lim_{x \to - \infty} F(x) = 0$
			\end{enumerate}
		\end{definition}
\end{document}























