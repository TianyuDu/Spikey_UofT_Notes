\documentclass[11pt]{article}

\title{ECO208 Macroeconomic Theory \\ Test 1 Review}
\author{Tianyu Du}
\date{\today}

\usepackage{amsmath}
\usepackage{amssymb}
\usepackage{pgfplots}
\usepackage{graphicx}
\usepackage{enumitem}
\usepackage{hyperref}
\usepackage{fancyhdr}
\usepackage{perpage}
\usepackage{float}

\lhead{Notes by T.Du}
\usepackage[
	type={CC},
	modifier={by-sa},
	version={3.0},
]
{doclicense}

\newcommand{\pd}[2]{\frac{\partial{#1}}{\partial{#2}}}


\begin{document}
\maketitle
\doclicenseThis
\tableofcontents

\section{Chapter 2: Measurements}
\subsection{Measuring GDP}
\paragraph{Three approaches}
\begin{enumerate}
	\item Expenditure Approach.
	\item Income Approach.
	\item Production (Value-added) Approach.
\end{enumerate}
\paragraph{Net Factor Payment(NFP)} Income paid towards domestic factors aboard minus income paid to foreign factors in board.
\[
	GNP = GDP + NFP
\]
\paragraph{Problem with GDP}
\begin{enumerate}
	\item Inequality.
	\item Non-market/ home production.
	\item Underground economy.
	\item Value added for government services. \underbar{Workaround: estimate with expenditure.}
\end{enumerate}

\subsection{Measuring Changes over Time}
\begin{enumerate}
	\item Constant price: take prices in \textbf{base year}.
	\item Chain-weighted method: calculate the growth rate with prices in difference year, then take the \textbf{geometric average}.
\end{enumerate}
\paragraph{Calculation} Let $p_n^i$ be the price of good $n$ in year $i$, and $q_n^i$ be the quantity of good $n$ produced in year $i$. Then
\[
	NGDP_t = \sum_{n=1}^N p_n^t q_n^t \text{ (nominal GDP of year $t$)}
\]
\[
	RGDP_t^b = \sum_{n=1}^N p_n^b q_n^t \text{ (real GDP of year $t$ with base year $b$)}
\]
For chain-weighted method, to calculate the growth rate between year $t$ and $t + 1$, let
\[
	1 + g_t = \frac{RGDP_{t+1}^t}{RGDP_{t}^t}
\]
\[
	1 + g_{t+1} = \frac{RGDP_{t+1}^{t+1}}{RGDP_{t}^{t+1}}
\]
Then, the chain-weighted growth rate, $g_c$ is 
\[
	1 + g_c = \sqrt{(1+g_t)(1+g_{t+1})}
\]
\paragraph{GDP Price Deflator}
\[
	Deflator = \frac{NGDP}{RGDP} \times 100
\]

\subsection{Saving and Investment}
\paragraph{Private Disposable Income}
\[
	Y^d = Y + NFP + TR + INT - T
\]
\paragraph{Private Saving}
\[
	S^{private} = Y^d - C = Y + NFP + TR + INT - T - C
\]
\paragraph{Public(Government Saving)}
\[
	S^{public} = T - TR - INT - G
\]
\paragraph{Total Saving}
\[
	S = S^{private} + S^{public} = Y - C - G + NFP = I + NX + NFP = I + CA
\]
Where $CA$ stands for \textbf{current account}, and $CA = NFP + NX$. Current account measures the net cash \emph{inflow} (from factor payment and product payment) into the country.
\subsection{Labor Market Measurement}
\paragraph{Measurements}
\[
	\text{Unemployment Rate} = \frac{\text{Unemployment}}{\text{Labor Force}}
\]
\[
	\text{Participation Rate} = \frac{\text{Labor Force}}{\text{Total Working Age Population}}
\]
\[
	\text{Employment-Population Ratio} = \frac{\text{Employment}}{\text{Total Working Age Population}}
\]

\section{Chapter 4: Consumer and Firm Behaviour in a One-period Model}
\subsection{Representative Agent: Consumer}
\subsubsection{Controls}
\paragraph{Physical Goods ($C$)} Assumed to be \underbar{normal good}, and abstracted to be a composite good, denoted as \textbf{aggregate consumption} ($C$).
\paragraph{Leisure ($\ell$)} Measured in hours (or other unit of time.)

\subsubsection{Assumptions}
\begin{enumerate}
	\item \textbf{Monotonicity} More is better.
	\item \textbf{Convexity} Diversity preferred.
\end{enumerate}

\subsubsection{Constraints}
\paragraph{Budget Constraint} Let $w$ denote the real wage rate, $\pi$ denote the dividend payment from. firms owned by household and $T$ be the lump sum tax collected by government. By Walras' Law, the constraint would holds as equality.
\begin{equation}
	C = w N^s + \pi - T
\end{equation}

\paragraph{Time Constraint} Let $h$ denote the total hours available in a single time period, let $N^s$ denote the time devoted to work and $\ell$ denote the time of leisure enjoyed.
\begin{equation}
	N^s + \ell = h
\end{equation}

\paragraph{Other (implicit) Constraints} Those constraints make sure that values of variable makes sense.
\begin{align}
	C \geq 0 \\
	0 \leq \ell \leq h \\
	0 \leq N^s \leq h
\end{align}

\subsubsection{Experiments}
\begin{enumerate}
	\item Change in \underbar{non-labor income} $\implies$ pure income effect.
	\item Change in \underbar{real wage rate} $\implies$ both income effect and substitution effect.
\end{enumerate}

\subsubsection{Constructing Labor Supply}
\paragraph{Labor Supply} Let $\ell^*(w, \cdot)$ denote the optimal level of leisure chosen by the consumer at real wage rate $w$ and other parameter given. Then by equation (2), the supply of labor could be constructed as $N^{s*}(w,\cdot) = h - \ell^*(w, \cdot)$.
\subsubsection{Formalizing Consumer's Optimization Problem}
\paragraph{Utility} Consider the log form of \textbf{Cobb-Douglas Utility Function}
\[
	u(c,\ell) = \log{c} + \eta \log{\ell},\ \eta > 0
\]
\paragraph{Optimization}
\begin{multline}
\\
	\max_{c,\ \ell} u(c,\ell) = \log{c} + \eta \log{\ell} \\
	s.t. \\
	c = (h-\ell)w+\pi-T \\
	c \geq 0 \\
	0 \leq \ell \leq h \\
\end{multline}
\paragraph{Solution} Set up the Lagrangian function and solve for the first order condition, we have
\[
	\mathcal{L}(c, \ell, \lambda) = \log{c} + \eta \log{\ell} + \lambda ((h-\ell)w+\pi-T - c)
\]
\begin{align}
	c^*(\cdot) = \frac{hw + \pi - T}{1 + \eta} \\
	\ell^* (\cdot) = \frac{hw + \pi - T}{w (1 + \frac{1}{\eta})}
\end{align}

\subsubsection{Comparative Statistics}
\paragraph{Lump Sum Tax ($T$)}
\begin{equation}
	\pd{c^*(\cdot)}{T} = -\frac{1}{1+\eta} < 0
\end{equation}
Therefore \underbar{negative} correlation.

\begin{equation}
	\pd{\ell^* (\cdot)}{T} = -\frac{1}{w(1+\frac{1}{\eta})} < 0
\end{equation}
Therefore \underbar{negative} correlation.

\paragraph{Real Wage Rate ($w$)}
\begin{equation}
	\pd{c^*(\cdot)}{w} = \frac{h}{1+\eta} > 0
\end{equation}
Therefore \underbar{positive} correlation.
\begin{equation}
	\pd{\ell^*(\cdot)}{w} = \frac{hw(1+ \frac{1}{\eta}) - (1+\frac{1}{\eta})(hw+\pi-T)}{w^2 (1+\frac{1}{\eta})^2} = -\frac{\pi - T}{w^2(1+\frac{1}{\eta})}
\end{equation}
The correlation is up to the sign of non-labor income ($\pi - T$).

\subsection{Representative Agent: Firm}
\subsubsection{Production Function}
\paragraph{Production Function} maps the inputs ($K, N$) to output ($Y$).
\[
	Y = z F(L, N^d),\ z > 0
\]
Where $z$ is the \textbf{total factor productivity (TFP)}

\subsubsection{Assumptions}
\paragraph{Constant Return to Scale (CRS)}
\begin{equation}
	F(t K, t N^d) = t F(K, N^d),\ \forall t > 0
\end{equation}
\paragraph{Increasing Return in Input}
\begin{equation}
	\pd{F(K, N^d)}{K} > 0 \land \pd{F(K, N^d)}{N^d} > 0
\end{equation}
\paragraph{Diminishing in Marginal Return}
\begin{equation}
	\frac{\partial^2F(K,N^d)}{\partial K^2} < 0 \land \frac{\partial^2F(K,N^d)}{\partial N^{d2}} < 0
\end{equation}

\paragraph{Marginal Product Increases in other Inputs}
\begin{equation}
	\frac{\partial^2F(K,N^d)}{\partial K \partial N^d} > 0 \land \frac{\partial^2F(K,N^d)}{\partial N^d \partial K} > 0
\end{equation}

\subsubsection{Optimization Problem}
\begin{multline}
	\\
	\max_{N^d} \{ z F(K, N^d) - w N^d \}
	\\
\end{multline}
\paragraph{Intuition} The optimal choice would be where $MP_N = w$, this means an extra unit of labor hired leads to negative profit change.
\subsubsection{Formalizing Firm's Optimization Problem}
\paragraph{Production Function} Take the \textbf{Cobb-Douglas Production Function} so that an interior solution for this optimization problem is guaranteed to be existing.
\begin{equation}
	Y = zF(K, N^d) = z K ^ \alpha {N^d}^{1-\alpha},\ \alpha \in (0,\ 1)
\end{equation}
\paragraph{Solution}
\begin{equation}
	{N^d}^*(\cdot) = (\frac{(1-\alpha)z K^\alpha}{w})^{\frac{1}{\alpha}}
\end{equation}

\subsubsection{Comparative Statistics}
\paragraph{Total Factor Productivity ($z$)}
\begin{equation}
	\pd{N^{d*}(\cdot)}{z} = \frac{1}{\alpha} z^{\frac{1}{\alpha} - 1} (\frac{(1-\alpha)K^\alpha}{w})^{\frac{1}{\alpha}} > 0
\end{equation}
\paragraph{Capital $K$}
\begin{equation}
	\pd{N^{d*}(\cdot)}{K} = (\frac{(1-\alpha)z}{w})^{\frac{1}{\alpha}} > 0
\end{equation}
\paragraph{Wage ($w$)}
\begin{equation}
	\pd{N^{d*}(\cdot)}{w} = -\frac{1}{\alpha} w ^ {-\frac{1+\alpha}{\alpha}} [(1-\alpha)z K^\alpha]^{\frac{1}{\alpha}} < 0
\end{equation}
\subsubsection{Adding Taxes}
\paragraph{Tax on output ($\tau$)}
\begin{equation}
	{N^d}^*(\cdot) = (\frac{(1-\alpha)(1-\tau)z K^\alpha}{w})^{\frac{1}{\alpha}}
\end{equation}

\paragraph{Tax on labor hired ($\tau_N$)}
\begin{equation}
	{N^d}^*(\cdot) = (\frac{(1-\alpha)z K^\alpha}{(1 + \tau_N)w})^{\frac{1}{\alpha}}
\end{equation}

\section{Chapter 5: General Equilibrium}
\subsection{Summary in Previous Lectures}
\paragraph{Competitive equilibrium} Each agent takes prices as given ($w$) when choosing $(N^D, N^S, C, \ell)$
\paragraph{Agent behaviours} The representative consumer and the representative firm solve their \textbf{optimization problems} given things they have no control over.

\subsection{One-Period General Equilibrium Model}
\subsubsection{Assumptions}
\paragraph{Closed economy} no trade with the outside world. ($X=M=NX=0$)
\paragraph{Static economy} one period model, therefore $S = I = 0$.
\subsubsection{Equilibrium Conditions}
\paragraph{Good market clearance} All physical consumption goods produced by the representative firm are consumed by the representative household or the government.
\[
C^* = Y^*
\]
\paragraph{Labor market clearance} The labour supplied by the household equals the labour demand of the firm.
\[
N^{S*} = N^{D*}
\]
\emph{The only price in this model is the \underbar{real wage rate $w$}.}

\subsubsection{With Government}
\begin{multline*}
	\\
	Y = C + G \text{ (product market clearance)} \\
	G = T \text{ (government budget balanced)} \\
\end{multline*}

\subsubsection{Variables}
\paragraph{Exogenous variables} $G, z, K$
\paragraph{Endogenous variables} $C, N^s, N^d, T, \pi, Y, w$

\paragraph{Competitive Equilibrium} is an allocation of goods and set of prices such that
\begin{enumerate}
	\item Agents take prices as given.
	\item Agents face a optimization problem.
	\item All markets clear.
\end{enumerate}

\end{document}
















