\documentclass{article}
\usepackage{spikey}
\usepackage{amsmath}
\usepackage{amssymb}
\usepackage{soul}
\usepackage{float}
\usepackage{graphicx}
\usepackage{hyperref}
\usepackage{xcolor}
\usepackage{chngcntr}
\usepackage{centernot}
\usepackage[shortlabels]{enumitem}
\usepackage[margin=1truein]{geometry}
\usepackage{tkz-graph}
\usepackage{dsfont}

\counterwithin{equation}{section}
\counterwithin{figure}{section}

\usepackage[
    type={CC},
    modifier={by-nc},
    version={4.0},
]{doclicense}

\title{ECO2020 Microeconomic Theory I (PhD) \\ \small Individual Decision Making, Market Equilibrium, Market Failure, and Other Topics.}
\date{\today}
\author{Tianyu Du}
\begin{document}
	\maketitle
	\doclicenseThis
	\begin{itemize}
		\item GitHub: \url{https://github.com/TianyuDu/Spikey_UofT_Notes}
		\item Website: \url{TianyuDu.com/notes}
	\end{itemize}
	\tableofcontents
	\newpage
	
	\section{Chapter 1. Preference and Choice}
		\begin{definition} \quad
			\begin{enumerate}[(i)]
				\item The \textbf{strict preference} relation, $\succ$, is defined by
					\begin{equation}
						x \succ y \iff x \succsim y \land \neg (y \succsim x)
					\end{equation}
				\item The \textbf{indifference} relation, $\sim$, is defined by
					\begin{equation}
						x \sim y \iff x \succsim y \land y \succsim x
					\end{equation}
			\end{enumerate}
		\end{definition}
	
		\begin{definition}[1.B.1]
			The preference relation $\succsim$ is \textbf{rational} if it possesses the following two properties
			\begin{enumerate}[(i)]
				\item \emph{Completeness} 
					\begin{equation}
						\forall x, y \in X,\ x \succsim y \lor y \succsim x
					\end{equation}
				\item \emph{Transitivity}
					\begin{equation}
						\forall x, y, z \in X,\ x \succsim y \land y \succsim z \implies x \succsim z
					\end{equation}
			\end{enumerate}
		\end{definition}
	
		\begin{proposition}[1.B.1]
			If $\succsim$ is rational, then
			\begin{enumerate}[(i)]
				\item 
			\end{enumerate}
		\end{proposition}
\end{document}




















































