\documentclass[11pt]{article}

\title{PHL245 Modern Symbolic Logic}

\author{Tianyu Du}
\date{\today}

\usepackage{spikey}
\usepackage{amsmath}
\usepackage{amssymb}
\usepackage{soul}
\usepackage{float}
\usepackage{graphicx}
\usepackage{hyperref}
\usepackage{xcolor}
\usepackage{chngcntr}
\usepackage{centernot}
\usepackage[shortlabels]{enumitem}
\usepackage[margin=1truein]{geometry}
\usepackage{sgame}
\usepackage{fancyvrb}

\counterwithin{equation}{section}
\counterwithin{figure}{section}

\usepackage[
    type={CC},
    modifier={by-nc},
    version={4.0},
]{doclicense}


\begin{document}
	\maketitle
	\doclicenseThis
	\texttt{Github Page:} \url{https://github.com/TianyuDu/Spikey_UofT_Notes}\\
	\texttt{Note Page:} \url{TianyuDu.com/notes}
	\section{Lecture 1}
	\section{Lecture 2}
		\begin{definition}
			An \textbf{argument} consists two parts, set of \textbf{premises} and \textbf{conclusions}.
			\begin{equation}
				\texttt{premise} \implies \texttt{conclusion}
			\end{equation}
			Where a \emph{premise} is a reason given to believe something, and the \emph{conclusion} is the thing the argument is supposed to cause one to believe.
		\end{definition}
		
		\begin{remark} Two types of arguments
			\begin{enumerate}
				\item Inductive Arguments.
				\item Deductive Arguments.
			\end{enumerate}
		\end{remark}
		
		\begin{remark}
			Deductive arguments are \textbf{truth-preserving}.
		\end{remark}
		
		\begin{definition}
			An argument is \textbf{valid} if and only if there's \emph{no logically possible situation} in which all of its premises are true and its conclusion is false.
		\end{definition}
		
		\begin{remark}[Vacuous]
			A valid argument could have false premise, or false conclusion.
		\end{remark}
		
		\begin{definition}
			An argument is \textbf{sound} if it's valid and its premises are all true.
		\end{definition}
		
		\begin{definition}
			An argument is \textbf{trivially valid} either
			\begin{enumerate}
				\item it has false premise,
				\item \ul{or}, its conclusion is unconditionally true (does not depend on the premises).
			\end{enumerate}
		\end{definition}
	
	\section{Lecture 3}
		\begin{definition}
			An \textbf{atomic sentence}(\textbf{proposition}) is the basic element of sentential logic, which has a \textbf{truth-value}.
		\end{definition}
		
		\begin{definition}
			A \textbf{molecular sentence} is an object with \textbf{truth-value} built up out of atomic sentences using connectives.
		\end{definition}
		
		\begin{definition}
			The \textbf{main connective} is the object determines the overall truth-value of the sentence.
		\end{definition}
		
		\begin{definition}
			\textbf{Parsing} is the process to figure out which symbol is the \textbf{main connective}.
		\end{definition}
		
		\begin{definition}
			A sentence is in \textbf{official notation} if and only if it is an atomic sentence, or can be constructed from atomic sentences using (i) negation and (ii) conditionals.
		\end{definition}
		
		\begin{definition}
			A sentence is in \textbf{informal notation} if you could turn it into official notation just by adding parentheses around it.
		\end{definition}
		
		\begin{definition}
			A sentence is \textbf{not well-formed} if it is not in official notation or informal notation.
		\end{definition}
		
		\begin{remark}Equivalences to Negation
			\begin{enumerate}[(i)]
				\item Not $X$.
				\item It's not the case $X$.
				\item It's failed to $X$.
			\end{enumerate}
		\end{remark}
		
		\begin{definition}
			A \textbf{conditional} connective takes the form
			\begin{equation}
				\texttt{antecedent} \implies \texttt{consequent}
			\end{equation}
			where \texttt{antecedent} is \emph{sufficient} for \texttt{consequent}, and, \texttt{consequent} is \emph{necessary} for \texttt{antecedent}.
		\end{definition}
		
		\begin{remark} Equivalences to Conditionals $P \implies Q$
			\begin{enumerate}[(i)]
				\item If $P$ then $Q$.
				\item Provided $P$, $Q$.
				\item Assuming that $P$, $Q$.
				\item Given that $P$, $Q$.
				\item \red{In case $P$, $Q$.}
				\item On the condition that $P$, $Q$>
			\end{enumerate}
		\end{remark}
		
	\section{Lecture 4}
		\begin{definition}
			A \textbf{derivation} is a method for proving that an argument is valid. It starts from some given \emph{premises}, leas by applications of \emph{rules}, to some \emph{conclusions}.
		\end{definition}
		
		\begin{rules}[Repetition (\texttt{R})]
			\begin{equation}
				P \therefore P
			\end{equation}
		\end{rules}
		
		\begin{rules}[Double Negation (\texttt{DN})]
			\begin{gather}
				P \therefore \sim \sim P \\
				\sim \sim P \therefore P
			\end{gather}
		\end{rules}
		
		\begin{rules}[Modus Ponens (\texttt{MP})]
			\begin{gather}
				P \implies Q.\ P \therefore Q
			\end{gather}
		\end{rules}
		
		\begin{rules}[Modus Tollens (\texttt{MT})]
			\begin{gather}
				P \implies Q.\ \sim Q \therefore \sim P
			\end{gather}
		\end{rules}
	
	\section{Lecture 5}
	\begin{remark}
		You must provide an exact match as input to a rule that involves negations.
		If a rule that involves negations produces a double negation \emph{as the major connective}, you may give yourself the unnegated version.
	\end{remark}
	
	\begin{definition}
		A \textbf{conditional derivation} is a proof that if we assume the antecedent, (\texttt{ass cd}) then the consequent will follow.
	\end{definition}
	\begin{figure}
		\centering
		\begin{BVerbatim}
Show P -> Q:
	P ass cd
	...
	Q cd
		\end{BVerbatim}
		\caption{conditional derivation usage}
	\end{figure}
	
	\begin{definition}
		An \textbf{indirect derivation} is a proof that if we assume the opposite of the conclusion (\texttt{ass id}), and show it leads to a contradiction.
	\end{definition}
	\begin{figure}
		\centering
		\begin{BVerbatim}
Show ...:
	...
	11. ~Q
	12. Q
	11 12 id
		\end{BVerbatim}
		\caption{indirect derivation usage}
	\end{figure}
	
	\begin{remark}
		When you close off a sub-derivation, you can’t use its contents in further derivations, except for the cancelled show line.
	\end{remark}
	
	\begin{remark}
		If we wish to use a line \emph{outside} the sub-derivation, we need to use the \emph{repetition rule} to bring the available line into the sub-derivation.
	\end{remark}
	
	\section{Lecture 6}
		\begin{definition}
			Connective \textbf{conjunction}
			\begin{equation}
				(P \land Q)
			\end{equation}
		\end{definition}
		
		\begin{definition}
			Connective \textbf{disjunction}
			\begin{equation}
				(P \lor Q)
			\end{equation}
		\end{definition}
		
		\begin{definition}
			Connective \textbf{bi-conditional}
			\begin{equation}
				(P \iff Q)
			\end{equation}
		\end{definition}
		
		\begin{remark}
			In general, for \emph{informal notations}, we assume the conditional or bi-conditional is the major connective, unless the brackets tell you otherwise
		\end{remark}
		\begin{example}
			\begin{gather}
				P \land Q \implies R \equiv (P \land Q) \implies R \\
				P \implies Q \lor R \equiv P \implies (Q \lor R)
			\end{gather}
		\end{example}
		
		\begin{remark}
			For informal notations for mixed disjunction and conjunction, the sentence is evaluated \emph{from left to right}.
		\end{remark}
		
		\begin{example}
			\begin{gather}
				P \lor Q \land R \equiv (P \lor Q) \land R
			\end{gather}
		\end{example}
		
		\begin{remark} Equivalences to AND
			\begin{enumerate}[(i)]
				\item $X$ and $Y$.
				\item $X$, but $Y$.
				\item $X$, although $Y$.
				\item $X$, even though $Y$.
				\item Even though $X$, $Y$.
				\item Despite $X$, $Y$.
			\end{enumerate}
		\end{remark}
		
		\begin{remark} Equivalences to OR
			\begin{enumerate}[(i)]
				\item $X$ or $Y$.
				\item Either $X$ or $Y$.
				\item $X$ \red{unless} $Y$.
			\end{enumerate}
		\end{remark}
		
		\begin{remark}
			\textbf{Not both} is equivalent to
			\begin{equation}
				\sim (P \land Q)
			\end{equation}
		\end{remark}
		
		\begin{remark}
			\textbf{Neither ... nor} is equivalent to
			\begin{equation}
				\sim (X \lor Y)
			\end{equation}
		\end{remark}
		
		\begin{remark} Equivalents to iff
			\begin{enumerate}[(i)]
				\item $X$ if and only if $Y$.
				\item $X$ exactly on the condition that $Y$.
				\item $X$ \red{\emph{just} in case} $Y$.
				\item $X$ is necessary and sufficient for $Y$.
			\end{enumerate}
		\end{remark}
		
		\begin{rules}[Negation Introduction (\texttt{dn})]
			\begin{equation}
				P \therefore \sim \sim P
			\end{equation}
		\end{rules}
		
		\begin{rules}[Negation Elimination (\texttt{dn})]
			\begin{equation}
				\sim \sim P \therefore P
			\end{equation}
		\end{rules}
		
		\begin{rules}[Simplification (\texttt{s})]
			\begin{gather}
				P \land Q \therefore P \\
				P \land Q \therefore Q
			\end{gather}
		\end{rules}
		
		\begin{rules}[Adjunction (\texttt{adj})]
			\begin{gather}
				P.\ Q \therefore P \land Q \\
				P.\ Q \therefore Q \land P
			\end{gather}
		\end{rules}
		
		\begin{remark}
			Order matters in conjunction,
			\begin{equation}
				P \land Q \centernot \equiv Q \land P
			\end{equation}
		\end{remark}
		
		\begin{rules}[Addition (\texttt{add})]
			\begin{equation}
				P \therefore P \lor Z \\
				P \therefore Z \lor P
			\end{equation}
		\end{rules}
		
		\begin{rules}[Modus Tollendo Ponens (\texttt{mtp})]
			\begin{gather}
				P \lor Q.\ \sim Q \therefore P \\
				P \lor Q.\ \sim P \therefore Q
			\end{gather}
		\end{rules}
		
		\begin{rules}[Bi-conditional to Conditional (\texttt{bc})]
			\begin{equation}
				P \iff Q \therefore P \implies Q \\
				P \iff Q \therefore Q \implies P
			\end{equation}
		\end{rules}
		
		\begin{rules}[Conditional to Bi-conditional (\texttt{cb})]
			\begin{gather}
				P \implies Q.\ Q \implies P \therefore P \iff Q \\
				P \implies Q.\ Q \implies P \therefore Q \iff P
			\end{gather}
		\end{rules}
	
	\section{Lecture 8}
		\begin{definition}
			A \textbf{theorem} of logic is a conclusion that can be proven \emph{using no premises}.
		\end{definition}
		
		\begin{rules}[Negation of Conditional (\texttt{NC})]
		\begin{gather}
			\sim (P \implies Q) \therefore P \land \sim Q \\
			P \land \sim Q \therefore \sim (P \implies Q)
		\end{gather}
		\end{rules}
		
		\begin{rules}[Conditional as Disjunction (\texttt{CDJ})]
			\begin{gather}
				P \implies Q \therefore \sim P \lor Q \\
				\sim P \lor Q \therefore P \implies Q \\
				\sim P \implies Q \therefore P \lor Q \\
				P \lor Q \therefore \sim P \implies Q
			\end{gather}
		\end{rules}
		
		\begin{rules}[Separation of Cases (\texttt{SC})]
			\begin{gather}
				P \lor Q.\ P \implies R.\ Q \implies R \therefore R \\
				P \implies R.\ \sim P \implies R \therefore R
			\end{gather}
		\end{rules}
		
		\begin{rules}[DeMorgans (\texttt{DM})]
			General form
			\begin{gather}
				\sim(\land_{i}P_i) \therefore \lor_i (\sim P_i) \\
				\sim(\lor_{i}P_i) \therefore \land_i (\sim P_i)
			\end{gather}
			Binary case
			\begin{gather}
				\sim (P \land Q) \therefore \sim P \lor \sim Q \\
				\sim (P \lor Q) \therefore \sim P \land \sim Q
			\end{gather}
			converse also works.
		\end{rules}
		
		\begin{rules}[Negation of Bi-conditional (\texttt{NB})]
			\begin{gather}
				\sim(P \iff Q) \therefore P \iff \sim Q\\
				P \iff \sim Q \therefore \sim (P \iff Q)
			\end{gather}
		\end{rules}
		
	\section{Lecture 9}
		\begin{definition}
			A \textbf{tautology} is a sentence that comes out true on truth value assignments. \emph{Major connective comes out as True for \ul{all rows}.}
		\end{definition}
		
		\begin{definition}
			A \textbf{contradiction} is a sentence that comes out as false on all truth value assignments. \emph{Major connective comes out as False for \ul{all rows}.}
		\end{definition}
		
		\begin{definition}
			A sentence is \textbf{contingent} if it is neither a tautology, nor a contradiction.
		\end{definition}
		
		\begin{definition}
			A set of sentences is \textbf{consistent} if they can all be true at once, otherwise, they are \textbf{contradictory}. \emph{If \ul{there is} a truth value assignment where all the sentences come out as true, then they are consistent.}
		\end{definition}
		
		\begin{remark}[Truth Tables for Determining Validity]
			A proposition is valid if and only if \ul{for all} rows where major connectives of the primes are all true, the  conclusion is also true.
		\end{remark}
\end{document}
























