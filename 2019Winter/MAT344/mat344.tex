\documentclass{article}
\usepackage{spikey}
\usepackage{amsmath}
\usepackage{amssymb}
\usepackage{soul}
\usepackage{float}
\usepackage{graphicx}
\usepackage{hyperref}
\usepackage{xcolor}
\usepackage{chngcntr}
\usepackage{centernot}
\usepackage[shortlabels]{enumitem}
\usepackage[margin=1truein]{geometry}
\usepackage{tkz-graph}
%\GraphInit[vstyle = Shade]

\counterwithin{equation}{section}
\counterwithin{figure}{section}

\def\Z{{\mathbb Z}}
\def\Q{{\mathbb Q}}
\def\R{{\mathbb R}}
\def\C{{\mathbb C}}

\newcommand{\bi}[2]{\begin{pmatrix}{#1}\\{#2}\end{pmatrix}}
%\counterwithin{equation}{section}


\title{MAT 344 Lecture Notes}
\date{\today}
\author{Tianyu Du}
\begin{document}
	\maketitle
	\tableofcontents
	\newpage
	
	\section{Strings, Sets, and Binomial Coefficients}
		\subsection{Strings and Sets}
			\begin{notation}
				Let $n \in \Z_{++}$, and we use $[n]$ to denote the $n$-element set $\{1,2,\dots,n\}$.
			\end{notation}
			
			\begin{definition}
				Let $X$ be a set, then an $X$-\textbf{string of length} (or a \textbf{word}/\textbf{array}) $n$ is a function $s:[n] \to X$, and $X$ is called the \textbf{alphabet} of the string, and each $x \in X$ is called a \textbf{character} or \text{letter}.
			\end{definition}
			
			\begin{remark}
				An $X$-string defined by $s: [n] \to X$ with length $n$ can be equivalently defined as a \textbf{sequence} consisting elements in $X$.
				\begin{equation}
					s(1)s(2)\dots s(n)
				\end{equation}
			\end{remark}
			
			\begin{definition}
				In the case $X = \{0, 1\}$, strings generated from $X$ are called \textbf{binary strings}. When $X = \{0,1,2\}$, strings are called \textbf{ternary strings}.
			\end{definition}
			
			\begin{definition}
				Let $X$ be a \emph{finite} set and let $n \in \Z_{++}$. An $X$-string $s = x_1 x_2 \dots x_n$ is a \textbf{permutation} of size $m$ if $x_i \neq x_j\ \forall x_i, x_j \in s$.
			\end{definition}
			
			\begin{proposition}
				If $X$ is an $m$-element set and $m \geq n \in \Z_{++}$, then the number of $X$-strings of length $n$ that are permutations is 
				\begin{equation}
					P(m, n) \equiv \frac{m!}{(m-n)!}
				\end{equation}
			\end{proposition}
			
			\begin{definition}
				Let $X$ be a \emph{finite} set and let $0 \leq k \leq |X|$. Then $S \subseteq X$ with $|S| = k$ is a \textbf{combination} of size $k$.
			\end{definition}
			
			\begin{proposition}
				Let $n, k \in \Z$ such that $0 \leq k \leq n$, then the number of combinations is
				\begin{equation}
					\bi{n}{k} \equiv \frac{P(n, k)}{n!} = \frac{n!}{k!(n-k)!}
				\end{equation}
			\end{proposition}
			
			\begin{proposition}
				For all integers $n$ and $k$ with $0 \leq k \leq n$
				\begin{equation}
					\bi{n}{k} = \bi{n}{n-k}
				\end{equation}
			\end{proposition}
			
			\begin{example}
				Binomial coefficients can be used to find the number of integer solutions of
				\begin{equation}
					\sum_{i=1}^k x_i \leq N
				\end{equation}
				given appropriate integers $k, N \in \Z$.
				\begin{enumerate}[(i)]
					\item $x_i > 0\ \forall i \in [k]$ and equality holds, then $C(N-1, k-1)$.
					\item $x_i \geq 0\ \forall i \in [k]$ and equality holds, then $C(N+k-1, k-1)$.\footnote{Simulate choosing $x_i + 1$ instead of $x_i$.}
					\item $x_i > 0\ \forall i \neq j, x_j = Z$ and equality holds, then $C(N-Z+k-2,k-2)$.
					\item $x_i > 0\ \forall i \in [k]$ and strict inequality holds, then $C(N-1, k)$.\footnote{Image there is a placeholder $x_{k+1} > 0$.}
					\item $x_i \geq 0\ \forall i \in [k]$ and strict inequality holds, then $C(N+k-1, k)$.
					\item $x_i \geq 0\ \forall i \in [k]$ and \emph{weak} inequality holds, $C(N+k, k)$ \footnote{This can be calculated by adding case (ii) and case (v) together, and apply Pascal's identity}.
					\begin{gather}
						\bi{N+k-1}{k-1} + \bi{N+k-1}{k} = \bi{N+k}{k}
					\end{gather}
				\end{enumerate}
			\end{example}
\end{document}





































