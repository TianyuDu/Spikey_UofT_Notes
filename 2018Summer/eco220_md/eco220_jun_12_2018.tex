\documentclass{article}

\title{ECO220 Lecture Notes}
\author{Tianyu Du}
\date{Jun. 12 2018}

\usepackage{amsmath}
\usepackage{amssymb}
\usepackage{mycommands}

\begin{document}
\maketitle

	\section{Chpater 9}
		\subsection{Applying Normal Distribution}

		\begin{theorem}
			Let $X \sim N(\mu, \sigma^2=\tx{variance})$, then
			\[
				z = \frac{x - \mu}{\sigma} \sim N(0, 1)
			\]
		\end{theorem}

		\paragraph{Example} Given $X \sim N(50, 10)$ then
		\newline
		(a) $P(X \geq 70) = P(\frac{x - \mu}{\sigma} \geq \frac{70 - 50}{10}) = P(z \geq 2)=.5 - 0.4772$
		\newline

		(b) $P(40 < x \leq 72) = P(\frac{40-50}{10} < X \leq \frac{72-50}{10})$
		\newline $= P(-1 < z \leq 2.2) = 0.3413 + 0.4861$
		\newline

		(c) $P(X \leq k) = 0.75$ find $k$.
			\begin{multline*}
				\emph{Soln.} \\
				P(X \leq k) = P(z \leq \frac{k-50}{10}) = 0.75 \\
				\implies \frac{k-50}{10} \approx 0.67 \\
				\implies k \approx 10 * 0.67 + 50 = 56.7\\
				\blacksquare
			\end{multline*}
		\newline

		(d) $P(k \leq X < 60) = 0.4$ find $k$.
			\begin{multline*}
				\emph{Soln.} \\
				P(\frac{k-50}{10} \leq \frac{x-\mu}{\sigma} < \frac{60-50}{10}) = .4 \\
				\implies P(\frac{k-50}{10} \leq z < 1) = .4 \\
				\tx{Notice that } P(0 \leq z < 1) = 0.3413 \\
				\implies P(\frac{k-50}{10} \leq z \leq 0) = .4 - 0.3413 = .0587 \\
				\implies \frac{k-50}{10} \approx -.15 \\
				\implies k \approx 48.5 \\
				\blacksquare
			\end{multline*}
		\newline

		(e) $P(40 \leq x < k) = 0.7$ find $k$.
			\begin{multline*}
				\emph{Soln.} \\
				P(40 \leq x < k) = 0.7 \implies P(-1 \leq z < \frac{k-50}{10}) = 0.7 \\
				\tx{Notice, } k > 0 \\
				\implies P(0 < z < \frac{k-50}{10}) = 0.2 \\
				\implies \frac{k-50}{10} \approx 0.52 \\
				\implies k \approx 55.2 \\
				\blacksquare
			\end{multline*}


		\paragraph{Example} Suppose GMAT score have an average 600, and std 100. Suppose GMAT score folow a \underbar{normal distribution}. What percent of a GMAT score are over 710?
			\begin{multline*}
				\emph{Soln.} \\
				X \sim N(600, 100) \\
				P(X > 710) = P(z > \frac{710-600}{100}) = P(z > 1.1) \\
				\tx{From normal table,} \\
				P(0 < z \leq 1.1) = 0.3643 \\
				\implies P(z > 1.1) = 0.5 - 0.3643 = 0.1357 \\
				\blacksquare
			\end{multline*}
		\newline

		(b) A university admits student to their MBA program when their GMAT score above 90 percentile. Find the minimum score to be admitted.
			\begin{multline*}
				\emph{Soln.} \\
				\tx{Let $k$ denote the minimum GMAT score to be admitted.} \\
				P(X < k) = 0.9 \\
				\implies P(z < \frac{k - 600}{100}) = 0.9 \\
				\implies P(0 < z < \frac{k - 600}{100}) = 0.4 \\
				\implies \frac{k-600}{100} \approx {= }1.28 \\
				\implies k \approx 728 \\
				\blacksquare
			\end{multline*}
		\newline
		(c) Three students are writing GMAT. What is the probability that at least two of them have GMAT scores over 700.
			\begin{multline*}
				\emph{Soln.} \\
				\tx{As binominal} \\
				p = P(X > 700) = P(z > 1) = .1587\\
				P(succ=2) = \begin{pmatrix} 3 \\ 2 \end{pmatrix} p^2 (1-p) \\
				P(succ=3) = p^3 \\
				P^* = P(succ=2) + P(succ=3) \\
				\blacksquare
			\end{multline*}
		\newline
		(d) Suppose 500 students are writing GMAT. What is the prob that at least 300 of them have GMAT scores over 620?
			\begin{multline*}
				\emph{Soln.} \\
				p = P(X > 620) = P(z > 0.2) = 0.5 - 0.0793 = 0.4207 \\
				P^* = \sum_{i=300}^{500}{\begin{pmatrix}500 \\ i\end{pmatrix}p^i(1-p)^{500-i}} \\
				\blacksquare
			\end{multline*}
			\underbar{Cannot find numerical answer by hand. Will approximate the answer using the normal distribution.}

		\subsection{Approximation Binominal Distribution Using Normal Distribution}
		\paragraph{Example} Let $X \sim B(n=500, p=0.4)$ then $\mu = \mathbb{E}(X) = np = 200$, and $\sigma^2 = npq = 120$ by property of Binomial distribution.
		\newline
		To find $P(X \geq 300)$ in Binomial distribution, approximate it with $P(X > 299.5)$ in Normal distribution, 
		\[
			P(X > 299.5) = P(z > \frac{299.5 - 200}{\sqrt{120}})
		\]
		Also, $P(X > 300)$ in Binomial distribution is approximated with $P(X > 300.5)$ in Normal distribution.
		\newline
		$P_B(X \leq 260) \approx P_N(X < 260.5)$
		\newline
		$P_B(250 < X \leq 320) \approx P_N(250.5 < X < 320.5)$


\end{document}















	