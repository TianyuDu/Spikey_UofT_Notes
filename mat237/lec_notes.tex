\documentclass{article}

\author{Tianyu Du}
\title{MAT237 Lecture Notes \\ \small A Compact Version of Notes by Tyler Holden}
\date{Jul. 2018}


\usepackage{spikey}
\usepackage{amsmath}
\usepackage{amssymb}
\usepackage[utf8]{inputenc}
\usepackage[english]{babel}


\begin{document}
	\maketitle
	\tableofcontents
	
	\section{The Topology of $\R ^n$}
	
	\begin{definition}
		A \textbf{set} is any collection of \underbar{distinct} objects.
	\end{definition}
	
	\begin{definition}
		Let $S$ be a set and $A$ and $B \subseteq S$, the binary operator \textbf{union} is defined as
		\[
			A \cup B = \{x \in S: x \in A \lor x \in B \}
		\]
	\end{definition}
	
	\begin{definition}
		Let $S$ be a set and $A$ and $B \subseteq S$, the binary operator \textbf{intersection} is defined as
		\[
			A \cap B = \{x \in S: x \in A \land x \in B \}
		\]
	\end{definition}
	
	\begin{definition}
		Let S be a set and $A \subseteq S$, then the \textbf{complement} of $A$ with respect to $S$ is defined as
		\[
			A^c = \{x \in S: x \notin A\}
		\] 
	\end{definition}
	
	\begin{definition}
		The \textbf{Cartesian Product} of two sets $A$ and $B$ is the collection of \underbar{ordered pairs}, one from $A$ and one from $B$, denoted as 
		\[
			A \times B = \{(a, b): a \in A \land b \in B \}
		\]	
	\end{definition}
	
	\begin{definition}
		Let $\trans{f}{A}{B}$ be a function, then
		
		1. If $U \subseteq A$ then we define the \textbf{image} of $U$ as
			\[
				f(U) = \{y \in B: \exists x \in U s.t.\ f(x) = y \} = \{f(x): x\in U\}
			\] 
		
		2. If $V \subseteq B$ then we define the \textbf{pre-image} of $V$ as 
			\[
				f^{-1}(V) = \{x \in A: f(x) \in V\}
			\]
	\end{definition}
	
	\underbar{STOP: PAGE10}
	
	
	
	
	
	
	
	
	
	
	
	
	
	
	
\end{document}