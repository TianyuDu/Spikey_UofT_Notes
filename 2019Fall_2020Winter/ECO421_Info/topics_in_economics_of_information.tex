\documentclass{article}
\usepackage{spikey}
\usepackage{amsmath}
\usepackage{mathrsfs}
\usepackage{amssymb}
\usepackage{soul}
\usepackage{float}
\usepackage{graphicx}
\usepackage{hyperref}
\usepackage{fancyhdr}
\usepackage{xcolor}
\usepackage{chngcntr}
\usepackage{centernot}
\usepackage[shortlabels]{enumitem}
\usepackage[margin=1truein]{geometry}
\usepackage{tkz-graph}
\usepackage{dsfont}
\usepackage{caption}
\usepackage{subcaption}

\usepackage{setspace}
\linespread{1.15}
\usepackage[margin=1truein]{geometry}

\counterwithin{equation}{section}
\counterwithin{figure}{section}

\pagestyle{fancy}
\lhead{Topics in Economics of Information}

\usepackage[
    type={CC},
    modifier={by-nc},
    version={4.0},
]{doclicense}

\title{ECO421: Topics in Economic of Information \\ \large Games with Incomplete Information}
\date{\today}
\author{Tianyu Du}
\begin{document}
    \maketitle
    \tableofcontents
    \newpage
    
    \section{Knowledge}
    \begin{notation}
    	Let $\Omega$ denote the all possible states of world, and let $A$ denote the set of all agents.
    \end{notation}
	
	\begin{definition}
		Every $E \subseteq \Omega$ is called an \textbf{event} or a type/piece of \textbf{information}.
	\end{definition}
	
    \begin{definition}
    	Let $E \subseteq \Omega$ be a piece of information and $i \in A$, and we say agent $i$ \textbf{knows} $E$ in state $\omega \in \Omega$ if this agent knows
    	\begin{enumerate}[(i)]
    		\item The true state of world $\omega^* \in E$;
    		\item but, all $\omega \in E$ are considered to be possible.
    	\end{enumerate}
    	Remark: the agent is certain about the true state $\omega^*$ if and only if $E = \{\omega^*\}$.
    \end{definition}
    
    \begin{definition}
    	For an agent $i \in A$, the \textbf{information structure} of this agent, $\mc{T}_i$, is a \emph{partition} of $\Omega$.
    \end{definition}
    
    \begin{notation}
    	Each element of the partition corresponds to one piece of information. \\
    	One may define a mapping $T_i(\cdot): \Omega \to \mc{T}_i$, such that for every $\omega \in \Omega$, let $T_i(\omega) \in \mc{T}_i$ denote the piece of information known by this agent in state $\omega$. \\
    	Equivalently, $T_i(\omega) \subseteq \Omega$ denotes the set of all states that $i$ considers possible given her information in state $\omega$.
    \end{notation}
    
    \begin{definition}
    	A \textbf{knowledge-based type space} is defined to be a tuple
    	\begin{align}
    		(\Omega, (\mc{T}_i)_{i \in A})
    	\end{align}
    \end{definition}
   
   	\begin{definition}[Equivalent Definition]
   		An agent $i$ is said to \textbf{know} information $E$ in state $\omega$ if and only if
   		\begin{align}
   			T_i(\omega) \subseteq E
   		\end{align}
   		That is, the true state $\omega^* \in T_i(\omega) \subseteq E$, so that $i$ believes $E$ to be certain.
   	\end{definition}
   	
   	\begin{definition}
   		Define $K_i(E)$ to be the set of states in which agent $i$ knows $E$, that is,
   		\begin{align}
   			K_i(E) := \{\omega \in \Omega: T_i(\omega) \subseteq E\}
   		\end{align}
   	\end{definition}
   	
   	\begin{definition}
   		Given a piece of information $F \subseteq \Omega$ characterizing that $\omega^* \in F$, the new \textbf{information structure updated by $F$} is defined to be
   		\begin{align}
   			(\Omega^F, (T^F_i(\cdot)))
   		\end{align}
   		where
   		\begin{align}
   			\Omega^F &= \Omega \cap F \\
   			T_i^F(\omega) &= T_i(\omega) \cap F\quad \forall \omega \in \Omega^F
   		\end{align}
   	\end{definition}
   	\begin{definition}
   		The set of states where $E$ is known given $F$ is 
   		\begin{align}
   			K_i(E|F) := \left\{
   			\omega \in \Omega^F: T_i^F(\omega) \subseteq E
   			\right\}
   		\end{align}
   	\end{definition}
   	
   	\begin{notation}
   		For $i, j \in A$, the following are equivalent:
   		\begin{enumerate}
   			\item states in which $i$ knows that player $j$ knows $E$;
   			\item states in which $i$ knows $K_j(E)$.
   		\end{enumerate}
   		This set of states can be denoted as
   		\begin{align}
   			K_i(K_j(E))
   		\end{align}
   	\end{notation}
   	
   	\begin{definition}
   		An event $E \subseteq \Omega$ is a \textbf{common knowledge} in state $\omega$ if
   		\begin{align}
   			\omega &\in \bigcap_{i \in A} K_i(E) \\
   			\omega &\in \bigcap_{j \in A} K_j \left(\bigcap_{i \in A} K_i(E) \right) \\
   			\omega &\in \bigcap_{k \in A} K_k \left[
	   			\bigcap_{j \in A} K_j \left(\bigcap_{i \in A} K_i(E) \right)
   			\right]
   		\end{align}
   	\end{definition}
\end{document}













