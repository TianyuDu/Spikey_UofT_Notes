\documentclass{article}
\usepackage{spikey}
\usepackage{amsmath}
\usepackage{mathrsfs}
\usepackage{amssymb}
\usepackage{soul}
\usepackage{float}
\usepackage{graphicx}
\usepackage{hyperref}
\usepackage{xcolor}
\usepackage{chngcntr}
\usepackage{centernot}
\usepackage[shortlabels]{enumitem}
\usepackage[margin=1truein]{geometry}
\usepackage{tkz-graph}
\usepackage{dsfont}

\counterwithin{equation}{section}
\counterwithin{figure}{section}

\usepackage[
    type={CC},
    modifier={by-nc},
    version={4.0},
]{doclicense}

\title{MAT395 Independent Reading in Mathematical Economics\\ \small Individual Decision Making, Market Equilibrium, Market Failure, and Other Topics.}
\date{\today}
\author{Tianyu Du}
\begin{document}
	\maketitle
	\doclicenseThis
	\begin{itemize}
		\item GitHub: \url{https://github.com/TianyuDu/Spikey_UofT_Notes}
		\item Website: \url{TianyuDu.com/notes}
	\end{itemize}
	\tableofcontents
	\newpage
	
	\section{Chapter 1. Preference and Choice}
	\subsection{Preference Relations}
	
		\begin{definition} \quad
			\begin{enumerate}[(i)]
				\item The \textbf{strict preference} relation, $\succ$, is defined by
					\begin{equation}
						x \succ y \iff x \pref y \land \neg (y \pref x)
					\end{equation}
				\item The \textbf{indifference} relation, $\sim$, is defined by
					\begin{equation}
						x \sim y \iff x \pref y \land y \pref x
					\end{equation}
			\end{enumerate}
		\end{definition}
	
		\begin{definition}[1.B.1]
			The preference relation $\pref$ is \textbf{rational} if it possesses the following two properties
			\begin{enumerate}[(i)]
				\item \emph{Completeness} 
					\begin{equation}
						\forall x, y \in X,\ x \pref y \lor y \pref x
					\end{equation}
				\item \emph{Transitivity}
					\begin{equation}
						\forall x, y, z \in X,\ x \pref y \land y \pref z \implies x \pref z
					\end{equation}
			\end{enumerate}
		\end{definition}
	
		\begin{proposition}[1.B.1]
			If $\pref$ is rational, then
			\begin{enumerate}[(i)]
				\item $\succ$ is both \textbf{reflexive} ($\neg\ x \succ x$) and \textbf{transitive} ($x \succ y \land y \succ z \implies x \succ z$);
				\item $\sim$ is both \textbf{reflexive} and \textbf{transitive};
				\item $x \succ y \pref z \implies x \succ z$.
			\end{enumerate}
		\end{proposition}
		
		\begin{example}
			Typical scenarios when transitivity of preference is violated:
			\begin{enumerate}[(i)]
				\item \emph{Just perceptible differences};
				\item \emph{Framing problem};
				\item \emph{Observed preference might from the result of the interaction of several more primitive rational preferences (Condorcet paradox)};
				\item \emph{Change of tastes}.
			\end{enumerate}
		\end{example}
		
		\begin{definition}[1.B.2]
			A function $u: X \to \R$ is a \textbf{utility function representing preference relation} $\pref$ if
			\begin{equation}
				\forall x, y \in X,\ x \pref y \iff u(x) \geq u(y)
			\end{equation}
		\end{definition}
		
		\begin{proposition}[1.B.2]
			If a preference relation $\pref$ can be represented by a utility function, then $\pref$ is rational.
		\end{proposition}
	
	\subsection{Choice Rules}
		\begin{definition}
			A \textbf{choice structure}, $(\mathscr{B}, C(\cdot))$, is a tuple consists of
			\begin{enumerate}[(i)]
				\item The collection of \textbf{budget sets} $\mathscr{B}$, which is a set of nonempty subsets of $X$.
				\item The \textbf{choice rule}, $C(B) \subset B$, is a \emph{correspondence} for every $B \subset \mathscr{B}$ denotes the individual's choice from among the alternatives in $B$. If $C(B)$ is not a singleton, it can be interpreted as the \emph{acceptable alternatives} in $B$, which the individual would actually chosen if the decision-making process is run repeatedly. 
			\end{enumerate}
		\end{definition}
		
		\begin{definition}[1.C.1]
			The choice structure $(\mathscr{B}, C(\cdot))$ satisfies the \textbf{weak axiom of revealed preference} if
			\begin{equation}
				\Big(\underbrace{
					\exists B \in \mathscr{B}\ s.t.\ x, y \in B \land x \in C(B)
					}_{\tx{$x \pref* y$ revealed.}}
				\Big)
				\implies 
				\Big(
					\forall B' \in \mathscr{B}\ s.t.\ x, y \in B',\ y \in C(B') \implies x \in C(B')
				\Big)
			\end{equation}
		\end{definition}
		
		\begin{definition}
			Given a choice structure $(\mathscr{B}, C(\cdot))$, the \textbf{revealed preference relation} $\pref^*$ is defined as
			\begin{equation}
				x \pref^* y \iff \exists B \in \mathscr{B}\ s.t.\ x, y \in B \land x \in C(B)
			\end{equation}
		\end{definition}
		
		\begin{remark}[Interpretation on the definition of WARP]
			If $x$ is \emph{revealed} at least as good as $y$, then $y$ cannot be revealed preferred to $x$.
		\end{remark}
	
	\subsection{The Relationship between Preference Relations and Choice Rules}
		\begin{definition}
			Given \ul{rational} preference relation $\pref$ on $X$, the \textbf{preference-maximizing choice rule} is defined as 
			\begin{equation}
				C^*(B, \pref) := \{x \in B: x \pref y\ \forall y \in B\}\ \forall B \in \mathscr{B}
			\end{equation}
			We say the \ul{rational} preference relation \textbf{generates} the choice structure $(\mathscr{B}, C^*(\cdot, \pref))$.
		\end{definition}
		
		\begin{assumption}
			Assume $C^*(B, \pref) \neq \varnothing$ for all $B \in \mathscr{B}$.
		\end{assumption}
		
		\begin{proposition}[1.D.1 (\hl{Rational $\to$ WARP})]
			Suppose that $\pref$ is a \ul{rational} preference relation. Then the choice structure generated by $\pref$, $(\mathscr{B}, C^*(\cdot, \pref))$, satisfies the weak axiom.
		\end{proposition}
		
		\begin{definition}[1.D.1]
			Given choice structure $(\mathscr{B}, C(\cdot))$, we say that the \ul{rational preference relation $\pref$ \textbf{rationalizes} $C(\cdot)$ relative to $\mathscr{B}$} if
			\begin{equation}
				C(B) = C^*(B, \pref)\ \forall B \in \mathscr{B}
			\end{equation}
			That is, \emph{$\pref$ generates the choice structure $(\mathscr{B}, C(\cdot))$}.
		\end{definition}
		
		\begin{remark}
			In general, for a given choice structure $(\mathscr{B}, C(\cdot))$, there may be more than one rational preference relation $\pref$ rationalizing it.
		\end{remark}
		
		\begin{proposition}[1.D.2 (\hl{WARP $\to$ Rational})]
			If $(\mathscr{B}, C(\cdot))$ is a choice structure such that
			\begin{enumerate}[(i)]
				\item The weak axiom is satisfied;
				\item $\mathscr{B}$ includes all subsets of $X$ up to three elements.
			\end{enumerate}
			Then there is a rational preference relation $\pref$ that rationalizes $C(\cdot)$ relative to $\mathscr{B}$.
		\end{proposition}
	
	\section{Chapter 2. Consumer Choice}
		\subsection{Commodities}
			\begin{definition}
				Assume the number of \textbf{commodities} is finite and equal to $L$. In general, a \textbf{commodity vector} or \textbf{commodity bundle} is an element in a \textbf{commodity space}, typically $\R^L$.
				\begin{equation}
					\vex := \begin{bmatrix}
						x_1 \\ \vdots \\ x_L
					\end{bmatrix} \in \R^L
				\end{equation}
 			\end{definition}
 			
 			\begin{remark}[Time Aggregation]
 				The time/location of commodity matters in some scenarios, and can be built into the definition of a commodity.
 			\end{remark}
 			
 			\begin{remark}
 				We should also note that in some contexts it becomes convenient, and even necessary, to expand the set of commodities to include goods and services that may potentially be available for purchase but are not actually so and even some that may be available by means other than market exchange.
 			\end{remark}
 		
 		\subsection{The Consumption Set}
 			\begin{definition}
 				The \textbf{consumption set} is a subset of the commodity space $\R^L$, denoted by $X \subset \R^L$, whose elements are the consumption bundles that the individual can conceivably consume given the physical constraints imposed by his environment.
 			\end{definition}
 			
 			\begin{assumption}
 				For simplicity, we assume the consumption set to be $\R_+^L$, which is \emph{convex}.
 				\begin{equation}
 					X := \R_+^L = \{\vex \in \R^L: x_{\ell} \geq 0,\ \forall \ell \in [L]\}
 				\end{equation}
 			\end{assumption}
 			
 		\subsection{Competitive Budgets}
 			\begin{definition}
 				A \textbf{price vector} is defined as
 				\begin{equation}
 					\vep := \begin{bmatrix} p_1 \\ \vdots \\ p_L \end{bmatrix} \in \R^L
 				\end{equation}
 				For simplicity, here we always assume 
 				\begin{enumerate}[(i)]
 					\item \emph{Positive price}: $\vep \gg \ve{0}$;
 					\item \emph{Price-taking assumption}: $\vep$ is beyond the influence of the consumer.
 				\end{enumerate}
 			\end{definition}
 			
 			\begin{definition}[2.D.1]
 				The \textbf{Walrasian}, or \textbf{competitive budget set} is defined as
 				\begin{equation}
 					B_{\vep, w} := \{
 						\vex \in \R^L_+: \vep \cdot \vex \leq w
 					\}
 				\end{equation}
 				where $w$ is the \emph{wealth} of consumer, and assumed to be positive.
 			\end{definition}
 			
 			\begin{definition}
 				The \textbf{consumer's problem} is choosing a consumption bundle $\vex \in B_{\vep, w}$, for each given $(\vep, w) \in \R^L_{++}$.
 			\end{definition}
 			
 			\begin{definition}
 				The set $\{\vex \in \R^L_+: \vep \cdot \vex = w\}$ is called the \textbf{budget hyperplane}.
 			\end{definition}
 			
 			\begin{proposition}
 				The price vector $\vep$ is orthogonal to the budget hyperplane.
 			\end{proposition}
 			
 			\begin{proposition}
 				The Walrasian budget set $B_{\vep, w}$ is a \emph{convex} set.
 			\end{proposition}
 		
 		\subsection{Demand Functions and Comparative Statics}
 			\begin{definition}
 				The consumer's \textbf{Walrasian demand correspondence} $x(\vep, w): \R_{++}^{L+1} \rightrightarrows \R_+^L$ assigns a \emph{set} of chosen consumption bundles for each price-wealth pair $(\vep, w)$. When $x(\vep, w)$ is single-valued, we refer to it as a \textbf{demand function}
 				\begin{equation}
 					\vex(\vep, w) = 
 					\begin{bmatrix}
 						x_1(\vep, w) \\
 						x_2(\vep, w) \\
 						\vdots \\
 						x_L(\vep, w)
 					\end{bmatrix}
 				\end{equation}
 			\end{definition}
 			
 			\begin{definition}[2.E.1]
 				The Walrasian demand correspondence $x(\vep, w): \R_{++}^{L+1} \rightrightarrows \R_+^L$ is \textbf{homogenous of degree zero} if 
 				\begin{equation}
 					x(\alpha \vep, \alpha w) = x(\vep, w)\ \forall (\vep, w, \alpha) \in \R_{++}^{L+2}
 				\end{equation}
 				Also note that
 				\begin{equation}
 					B_{\vep, w} = B_{\alpha \vep, \alpha w}\ \forall (\vep, w, \alpha) \in \R_{++}^{L+2}
 				\end{equation}
 			\end{definition}
 			
 			\begin{definition}[2.E.2]
 				The Walrasian demand correspondence $x(\vep, w)$ satisfies \textbf{Walras' law} if
 				\begin{equation}
 					\forall (\vep, w) \gg \ve{0},\ \forall \vex \in x(\vep, w),\ \vep \cdot \vex = w
 				\end{equation}
 			\end{definition}
 			
 			\begin{assumption}
 				For simplicity, we assume $x(\vep, w)$ is always \emph{single-valued, continuous and differentiable}.
 			\end{assumption}
 			
 			\begin{proposition}
 				The \textbf{family of Walrasian budget sets} defined as
 				\begin{equation}
 					\mathscr{B}^{\mathscr{W}} := \{B_{\vep, w}: {\vep, w} \gg \ve{0}\}
 				\end{equation}
 				altogether with Walrasian demand homogeneous to degree zero forms a \emph{choice structure}
 				\begin{equation}
 					(\mathscr{B}^{\mathscr{W}}, x(\cdot))
 				\end{equation} 
 			\end{proposition}
 		
 			\begin{definition}
 				For fixed prices $\overline{\vep} \in \R_{++}^L$, the function of wealth $\vex(\overline{\vep}, w)$ is called consumer's \textbf{Engel function}. Its image in $\R^L_+$,
 				\begin{equation}
 					E_{\overline{\vep}} := \{\vex(\overline{\vep}, w): w \in \R_{++}\} \subset \R^L_+
 				\end{equation}
 				is defined as the \textbf{wealth expansion path}.
 			\end{definition}
 			
 			\begin{definition}
 				Given $(\vep, w)$, the \textbf{wealth effect} is defined as
 				\begin{equation}
 					D_w \vex(\vep, w) 
 					= \begin{bmatrix}
 						\pd{x_1(\vep, w)}{w} \\
 						\pd{x_2(\vep, w)}{w} \\
 						\vdots \\
 						\pd{x_L(\vep, w)}{w}
 					\end{bmatrix} \in \R^L
 				\end{equation}
 				For the $\ell$-th commodity, it's called \textbf{normal} at $(\vep, w)$ if $\pd{x_{\ell}(\vep, w)}{w} \geq 0$, and \textbf{inferior} otherwise. And the $\ell$-th commodity is normal/inferior if its normal/inferior every where in $\R_{++}^{L+1}$. 
 			\end{definition}
 			
 			\begin{definition}
 				The \textbf{offer curve} is defined as the locus
 				\begin{equation}
 					\{\vex(\vep, w): p_j > 0\}
 				\end{equation}
 				for any chosen $j$.
 			\end{definition}
 			
 			\begin{definition}
 				Good $\ell$ is said to be a \textbf{Giffen good} at $(\vep, w)$ if
 				\begin{equation}
 					\pd{x_\ell(\vep, w)}{p_\ell} > 0
 				\end{equation}
 			\end{definition}
 			
 			\begin{definition}
 				The \textbf{price effects} at $(\vep, w)$ is defined as
 				\begin{equation}
 					D_{\vep} (\vep, w) = 
 					\begin{bmatrix}
 						\pd{x_1(\vep, w)}{p_1} & \cdots & \pd{x_1(\vep, w)}{p_L} \\
 						& \ddots & \\
 						\pd{x_L(\vep, w)}{p_1} & \cdots & \pd{x_L(\vep, w)}{p_L}
 					\end{bmatrix}
 				\end{equation}
 			\end{definition}
 			
 			\begin{proposition}[2.E.1]
 				If the Walrasian demand function $x(\vep, w)$ is homogenous of degree zero, then for all $\vep$ and $w$, then
 				\begin{equation}
 					\sum_{k=1}^{L} \frac{\partial x_{\ell}(\vep, w)}{\partial p_{k}} p_{k}+\frac{\partial x_{\ell}(\vep, w)}{\partial w} w=0 \text { for } \ell=1, \ldots, L
 				\end{equation}
 				Equivalently,
 				\begin{equation}
 					D_{\vep} \vex (\vep, w)\ \vep + D_w \vex(\vep, w)\ w = \ve{0}
 				\end{equation}
 				\begin{proof}
 					Apply \emph{Euler's theorem} on homogenous functions to each component $x_\ell$.
 					\begin{gather}
 						\underbrace{D_{(\vep, w)} \vex(\vep, w)}_{L \times (L+1)} \cdot \underbrace{(\vep, w)}_{(L+1) \times 1} = 0\ \vex(\vep, w) = \ve{0} \\
 						\implies [\underbrace{D_{\vep} (\vep, w)}_{L \times L} | \underbrace{D_w \vex(\vep, w)}_{L \times 1}] \cdot (\vep, w) = D_{\vep} (\vep, w)\ \vep + D_w \vex(\vep, w)\ w = \ve{0}
 					\end{gather}
 				\end{proof}
 			\end{proposition}
 			
 			\begin{definition}
 				The \textbf{elasticities of demand $\ell$ with respect to price $k$ and wealth} is defined as 
 				\begin{gather}
 					\varepsilon_{\ell, k} (\vep, w) := \pd{x_\ell (\vep, w)}{p_k} \frac{p_l}{x_\ell (\vep, w)}\\
 					\varepsilon_{\ell, w} (\vep, w) := \pd{x_\ell (\vep, w)}{w} \frac{w}{x_\ell (\vep, w)}
 				\end{gather}
 			\end{definition}
 			
  			\begin{corollary}
  				Dividing both sides of the equality in proposition (2.E.1) by $x_{\ell}$ gives
 				\begin{equation}
 					\sum_{k=1}^L \varepsilon_{\ell, k}(\vep, w) + \varepsilon_{\ell, w}(\vep, w) = 0\ \forall \ell \in \{1, \ldots, L\}
 				\end{equation}
 			\end{corollary}
 			
 			\begin{proposition}[2.E.2 Cournot Aggregation]
 				If the Walrasian demand function $x(\vep, w)$ satisfies \emph{Walras' law}, then for every $(\vep, w)$,
 				\begin{equation}
 					\sum_{\ell=1}^{L} p_{\ell} \frac{\partial x_{\ell}(\vep, w)}{\partial p_{k}}+x_{k}(\vep, w)= 0 \quad \text { for } k=1, \ldots, L
 				\end{equation}
 				Equivalently,
 				\begin{equation}
 					\vep^T D_{\vep} \vex (\vep, w) + \vex(\vep, w)^T = \ve{0}^T
 				\end{equation}
 				\begin{proof}
 					Differentiate both sides of Walras' law identity $\vep^T \vex = w$ with respect to $\vep$.
 				\end{proof}
 			\end{proposition}
 			
 			\begin{proposition}[2.E.3. Engel Aggregation]
 				If the Walrasian demand function $x(\vep, w)$ satisfies \emph{Walras' law}, then for every $(\vep, w)$,
 				\begin{equation}
 					\sum_{\ell=1}^{L} p_{\ell} \frac{\partial x_{\ell}(\vep, w)}{\partial w}=1
 				\end{equation}
 				or equivalently
 				\begin{equation}
 					\vep \cdot D_{w} x(\vep, w) = 1
 				\end{equation}
 				\begin{proof}
 					Differentiate both sides of Walras' law identity $\vep^T \vex = w$ with respect to $w$.
 				\end{proof}
 			\end{proposition}
 		
 			\begin{proposition}[Exer. 2.E.2]
 				\begin{gather}
 					\sum_{\ell=1}^{L} b_{\ell}(\vep, w) \varepsilon_{\ell k}(\vep, w)+b_{k}(\vep, w)=0
 				\end{gather}
 				and
 				\begin{gather}
 					\sum_{\ell=1}^{L} b_{\ell}(\vep, w) \varepsilon_{\ell w}(\vep, w)=1
 				\end{gather}
 				where $b_\ell := \frac{x_\ell p_\ell}{w}$ is defined to be the portion of wealth spent on commodity $\ell$.
 			\end{proposition}
 			
 		\subsection{The Weak Axiom of Revealed Preference and the Law of Demand}
 			\begin{assumption}
 				In the section, we assume $\vex(\vep, w)$ is 
 				\begin{enumerate}[(i)]
 					\item Single-valued;
 					\item homogeneous to degree zero;
 					\item satisfies Walras' law.
 				\end{enumerate}
 			\end{assumption}
 			
 			\begin{definition}[2.F.1]
 				The Walrasian demand function $\vex(\vep, w)$ satisfies the \textbf{weak axiom of revealed preference} if for every two $(\vep, w), (\vep', w') \in \R_{++}^{L+1}$,
 				\begin{gather}
 					\underbrace{\vep \cdot \vex(\vep', w') \leq w
 					\land \vex(\vep, w) \neq \vex(\vep', w')}_{\tx{revealed: } \vex(\vep, w) \succ^* \vex(\vep', w')}
 					\implies \vep' \cdot \vex(\vep, w) > w'
 				\end{gather}
 				Equivalently,
 				\begin{equation}
 					\vex(\vep', w') \in B_{\vep, w} \land \vex(\vep', w') \centernot \in C(B_{\vep, w}) \implies \vex(\vep, w) \centernot \in C(B_{\vep', w'})
 				\end{equation}
 			\end{definition}
 			
 			\begin{corollary}[Equivalent Definition ]
 				The weak axiom says, given our assumptions and $\vex(\vep_1, w_1) \neq \vex(\vep_2, w_2)$, we \ul{cannot} have both 
 				\begin{equation}
 					\vex(\vep_1, w_1) \in B_{\vep_2, w_2} \land \vex(\vep_2, w_2) \in B_{\vep_1, w_1}
 				\end{equation}
 			\end{corollary}
 			
 			\begin{definition}
 				A price change $\Delta \vep$ is a \textbf{Slutsky compensated price change} if the consumer is given a \textbf{Slutsky wealth compensation} with amount
 				\begin{equation}
 					\Delta w = \Delta \vep \cdot \vex(\vep, w)
 				\end{equation}
 				such that the consumer's initial consumption is just affordable at the new price.
 			\end{definition}
 			
 			\begin{proposition}[2.F.1]
 				Suppose that the Walrasian demand function $\vex(\vep', w')$ is homogenous of degree zero and satisfies Walras' law. Then $\vex(\vep', w')$ satisfies the weak axiom \ul{if and only if} the following property holds: \\
 				For any \emph{compensated price change} from $(\vep, w)$ to $(\vep', w' := \vep' \cdot \vex(\vep, w))$,
 				\begin{equation}
 					\Delta \vep \cdot \Delta \vex \leq 0
 				\end{equation}
 				with strict inequality whenever $\vex(\vep, w) \neq \vex(\vep', w')$.
 			\end{proposition}
 			
 			\begin{corollary}[Compensated Law of Demand]
 				$\Delta \vep \cdot \Delta \vex \leq 0$ says demand and price move in opposite directions, \emph{under Slutsky compensation}.
 			\end{corollary}
 			
 			\begin{definition}
 				At infinitesimal price change, the Slutsky compensation can be written as 
 				\begin{equation}
 					dw = \vex(\vep, w) \cdot d\vep
 				\end{equation}
 				and the compensated law of demand becomes
 				\begin{equation}
 					d\vep \cdot d\vex \leq 0
 				\end{equation}
 				Then the total derivative of $\vex$ is 
 				\begin{align}
 					d\vex &= D_\vep\vex(\vep, w)\ d\vep + D_w \vex(\vep, w)\ dw \\
 					&= D_\vep\vex(\vep, w)\ d\vep + D_w \vex (\vep, w)\ [\vex(\vep, w) \cdot d\vep] \\
 					&= [\underbrace{D_\vep\vex(\vep, w)}_{L \times L} + \underbrace{D_w \vex (\vep, w)}_{L \times 1} \underbrace{\vex(\vep, w)^T}_{1 \times L}] d\vep \\
 					&\implies d\vep^T 
 					\underbrace{[D_\vep\vex(\vep, w) + D_w \vex (\vep, w) \vex(\vep, w)^T]}_{L \times L} d\vep \leq 0
 				\end{align}
 				and the \textbf{Slutsky/substitution matrix} is defined as
 				\begin{gather}
 					S(\vep, w) := [D_\vep\vex(\vep, w) + D_w \vex (\vep, w) \vex(\vep, w)^T] \\
 					s_{\ell k} = \underbrace{\pd{x_\ell(\vep, w)}{p_k}}_{\tx{total effect}} + \underbrace{\pd{x_\ell(\vep, w)}{w} x_k (p, w)}_{\tx{wealth effect}}
 				\end{gather}
 				where $s_{\ell k}$ is the \textbf{substitution effect}.
 			\end{definition}
 			
 			\begin{remark}
 				The above identity (Slutsky equation) suggests the total impact of price change in $p_k$ on demand for $x_{\ell}$ can be decomposed into two portions, substitution effect and income effect.
 			\end{remark}
 			
  			\begin{corollary}[Slutsky Equation]
 				\begin{equation}
 					\frac{\partial x_{i}(\mathbf{p}, w)}{\partial p_{j}}=\frac{\partial h_{i}(\mathbf{p}, u)}{\partial p_{j}}-\frac{\partial x_{i}(\mathbf{p}, w)}{\partial w} x_{j}(\mathbf{p}, w)
 				\end{equation}
 			\end{corollary}
 			
 			\begin{remark}
 				Consider the scenario when only $p_k$ changes, with Slutsky compensation, consumer's wealth changes by $dw = x_k(\vep, w) dp_k$. So the wealth effect on $x_\ell$ is $\pd{x_\ell}{w} dw = \pd{x_\ell}{w} x_k(\vep, w) dp_k$.
 			\end{remark}
 			
 			\begin{proposition}[2.F.2]
 				If a differentiable Walrasian demand function $\vex(\vep, w)$ satisfies Walras' law, homogeneity of degree zero, and the weak axiom, then at any $(\vep, w)$, the Slutsky matrix $S(\vep, w)$ is negative semi-definite.
 			\end{proposition}
 			
 			\begin{corollary}
 				Given $S(\vep, w)$ is negative semi-definite, we have
 				\begin{align}
 					\ve{e}_{\ell}^T S(\vep, w) \ve{e}_{\ell} \leq 0\ \forall \ell \in \{1, \ldots, L\} \\
 					\implies s_{\ell \ell} \leq 0\ \forall \ell \in \{1, \ldots, L\}
 				\end{align}
 				which suggests the \emph{substitution effect of good $\ell$ with respect to its own price is always negative}.
 			\end{corollary}
 			
 			\begin{remark}
 				Proposition 2.F.2 does \emph{not} imply, in general, that the matrix $S(\vep, w)$ is symmetric.
 			\end{remark}
 			
 			\begin{proposition}[2.F.3]
 				Suppose that the Walrasian demand function $\vex(\vep, w)$ is differentiable, homogenous of degree zero, and satisfies Walras' law. Then for every $(\vep, w)$
 				\begin{equation}
 					\vep^T S(\vep, w) = \ve{0} \land S(\vep, w) \vep = \ve{0}
 				\end{equation}
 				\begin{proof}
 					By propositions 2.E.1 to 2.E.3.
 				\end{proof}
 			\end{proposition}
 	
 	\section{Chapter 3. Classical Demand Theory}
 		\subsection{Preference Relations: Basic Properties}
 			\begin{definition}[3.B.1]
 				The preference relation $\pref$ on $X$ is \textbf{rational} if it possesses the following two properties
 				\begin{enumerate}[(i)]
 					\item \emph{Completeness.} $\forall \vex, \vey \in X,\ \vex \pref \vey \lor \vey \pref \vex$;
 					\item \emph{Transitivity.} $\forall \vex, \vey, \vez \in X,\ \vex \pref \vey \land \vey \pref \vez \implies \vex \pref \vez$.
 				\end{enumerate}
 			\end{definition}
 			
 			\subsubsection{Desirability Assumptions}
 			
 			\begin{definition}[3.B.2]
 				The preference relation $\pref$ on $X$ is \textbf{monotone} if $\vex \in X$ and $\vey \gg \vex \implies \vey \succ \vex$; It is \textbf{strongly monotone} if $\vey \geq \vex \land \vex \neq \vey$
 			\end{definition}
 			
 			\begin{remark}
 				If $\pref$ is monotone, we may have indifference with respect to an increase in the amount of some but not all commodities.
 			\end{remark}
 			
 			\begin{definition}[3.B.3]
 				A preference relation $\pref$ on $X$ is \textbf{locally nonsatiated} if
 				\begin{equation}
 					\forall \vex \in X, \varepsilon > 0,\ \exists \vey \in \overline{\mc{B}}(\vex, \varepsilon) \cap X\ s.t.\ \vey \succ \vex
 				\end{equation}
 			\end{definition}
 			
 			\begin{remark}
 				Local nonsatiation rules out the extreme situation in which all commodities are bads, since in that case no consumption at all (the point $\vex = \ve{0}$) would be a satiation point.
 			\end{remark}
 			
 			\begin{proposition}[Exercise 3.B.1]
 				\begin{equation}
 					\tx{Strongly Monotone} \implies \tx{Monotone} \implies \tx{Locally Non-satiation}
 				\end{equation}
 			\end{proposition}
 			
 			\begin{definition}
 				The \textbf{indifference set} containing point $\vex$ is defined as $\{\vey \in X: \vex \sim \vey \}$. The \textbf{upper contour set} of bundle $\vex$ is $\{\vey \in X: \vey \pref \vex\}$. The \textbf{lower contour set} of $\vex$ is defined as $\{\vey \in X: \vex \pref \vey\}$.
 			\end{definition}
 			
 			\begin{remark}[Implication of Local Nonsatiation]
 				One implication of local nonsatiation (and, hence, of monotonicity) is that it rules out "thick" indifference sets.
 			\end{remark}
 			
 			\subsubsection{Convexity Assumptions}
 			
 			\begin{definition}[3.B.4]
 				The preference relation $\pref$ on $X$ is \textbf{convex} if for every $\vex \in X$, the upper contour set if $\vex$ is convex.
 				\begin{equation}
 					\forall \vex, \vey, \vez \in X,\ \vey \pref \vex \land \vez \pref \vex \implies \alpha \vey + (1-\alpha) \vez \pref \vex\ \forall \alpha \in [0, 1]
 				\end{equation}
 			\end{definition}
 			
 			\begin{remark}[Implication of Convexity]
 				Convexity can also be viewed as the formal expression of a basic inclination of economic agents for diversification.
 			\end{remark}
 			
 			\begin{remark}
 				The convex assumption can hold only if $X$ is convex.
 			\end{remark}
 			
 			\begin{definition}[3.B.5]
 				The preference relation $\pref$ on $X$ is \textbf{strictly convex} if
 				\begin{equation}
 					\forall \vex, \vey, \vez \in X,\ \vey \succ \vex \land \vez \succ \vex \land \vey \neq \vez \implies \alpha \vey + (1-\alpha) \vez \succ \vex\ \forall \alpha \in (0, 1)
 				\end{equation}
 			\end{definition}
 			
 			\begin{definition}[3.B.6]
 				A \ul{monotone} preference relation $\pref$ on $X = \R^L_+$ is \textbf{homothetic} if 
 				\begin{equation}
 					\forall \vex, \vey \in X,\ \vex \sim \vey \implies \alpha \vex \sim \alpha \vey,\ \forall \alpha \in \R_+
 				\end{equation}
 			\end{definition}
 			
 			\begin{definition}[3.B.7]
 				The preference relation $\pref$ on $X = (-\infty, \infty) \times \R^{L-1}_+$ is \textbf{quasilinear} with respect to commodity 1 (the \textbf{numeraire} commodity) if $\forall \vex, \vey \in X$
 				\begin{enumerate}[(i)]
 					\item $\vex \sim \vey \implies \vex + \alpha \ve{e}_1 \sim \vey + \alpha \ve{e}_1\ \forall \alpha \in \R$;
 					\item \emph{Good 1 is desirable}: $\forall \vex \in X, \alpha \in \R_{++},\ \vex + \alpha \ve{e}_1 \succ \vex$.
 				\end{enumerate}
 			\end{definition}
\end{document}



















































