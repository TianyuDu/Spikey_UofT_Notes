\documentclass{article}
\usepackage{amsmath}
\date{\today}
\author{Tianyu Du}
\title{Notes on ANT100: Introduction to evolutionary anthropology}
\begin{document}
	\maketitle
	\tableofcontents
	\section{Introduction to Evolutionary Anthropology}
	\paragraph{Anthropology} The global and holistic study of human culture and biology.
	\subsection{What Do Evolutionary Anthropologists Study?}
	\begin{itemize}
		\item Primatology
			\begin{itemize}
				\item The scientific study of our closest extant biological relatives: non-human primate species.
			\end{itemize}
		\item Paleoanthropology
			\begin{itemize}
				\item The multidisciplinary study of the biological evolution of humans and non-human primates.
			\end{itemize}
		\item Human Variation
			\begin{itemize}
				\item Study of human variation to determine spatial and temporal variation in human features.
			\end{itemize}
		\item Medical Anthropology
			\begin{itemize}
				\item The study of how social, environmental, and biological factors influence health, and illness of individuals at the community, regional, national, and global levels, is a recent addition to evolutionary anthropology.
			\end{itemize}
		\item Forensic Anthropology
			\begin{itemize}
				\item Focuses \emph{only} on the skeletal remains of humans.
			\end{itemize}
	\end{itemize}
	\subsection{How Do Evolutionary Anthropologists Conduct Their Research?}
	\paragraph{}Three types of research: \emph{descriptive}, \emph{causal} and \emph{applied}.
	\begin{itemize}
		\item \textbf{Descriptive research} involves \emph{collecting data} about the study subjects or objects.
		\item \textbf{Causal research} involves looking for one thing that \emph{causes} another thing to happen or change.
		\item \textbf{Applied research}, in which, a scientist determines the means by which a specific, recognized need can be met.
	\end{itemize}
	\subsubsection{What's a theory?}
	\paragraph{}A \textbf{scientific theory} is a well-substantiated explanation of some aspect of the natural world that incorporates facts, laws, predictions, and tested hypotheses.
	\subsubsection{What's a Hypothesis?}
	\paragraph{}A \textbf{hypothesis} is a \emph{testable} statement about the natural world that a researcher uses to build \emph{inferences} and \emph{explanations}. A hypothesis must be \textbf{falsifiable}.
	\subsubsection{The Scientific Method}
	\paragraph{Sequences of scientific methods:}
	\begin{enumerate}
		\item \textbf{Observation} of the phenomena.
		\item Formulation of a \textbf{hypotheses} concerning the phenomena.
		\item Development of \textbf{method} to test the validity of the hypothesis.
		\item \textbf{Experimentation}.
		\item Draw the \textbf{Conclusion}.
	\end{enumerate}
	\subsection{Development of Evolutionary Concepts}
	\subsubsection{Historical Contributors}
	\paragraph{Carl Linnaeus} \emph{Father of modern \textbf{taxonomy}}.
	\paragraph{Georges-Louis Leclerc} A naturalist. Original idea: \emph{species changed and evolved after they moved away from the place where they were created}.
	\paragraph{Jean-Baptiste Lamarck}
	\paragraph{Georges Cuvier} helped establish the scientific disciplines of \textbf{comparative anatomy} and \textbf{palaeontology}.
	\paragraph{James Hutton} contributed to founds of \textbf{geology} as a science.
	\paragraph{Carles Lyell} 
\end{document}





























