\documentclass{article}

\author{Tianyu Du}
\title{MAT237 Lecture Notes \\ \small A Compact Version of Notes by Tyler Holden}
\date{Jul. 2018}


\usepackage{spikey}
\usepackage{amsmath}
\usepackage{amssymb}
\usepackage{color}


\begin{document}
	\maketitle
	\tableofcontents
	
	\section{The Topology of $\R ^n$}
	\section{Sets and Notataion}
	
	\begin{definition}
		A \textbf{set} is any collection of \underbar{distinct} objects.
	\end{definition}
	
	\begin{definition}
		Let $S$ be a set and $A$ and $B \subseteq S$, the binary operator \textbf{union} is defined as
		\[
			A \cup B = \{x \in S: x \in A \lor x \in B \}
		\]
	\end{definition}
	
	\begin{definition}
		Let $S$ be a set and $A$ and $B \subseteq S$, the binary operator \textbf{intersection} is defined as
		\[
			A \cap B = \{x \in S: x \in A \land x \in B \}
		\]
	\end{definition}
	
	\begin{definition}
		Let S be a set and $A \subseteq S$, then the \textbf{complement} of $A$ with respect to $S$ is defined as
		\[
			A^c = \{x \in S: x \notin A\}
		\] 
	\end{definition}
	
	\begin{definition}
		The \textbf{Cartesian Product} of two sets $A$ and $B$ is the collection of \underbar{ordered pairs}, one from $A$ and one from $B$, denoted as 
		\[
			A \times B = \{(a, b): a \in A \land b \in B \}
		\]	
	\end{definition}
	
	\begin{definition}
		Let $\trans{f}{A}{B}$ be a function, then
		
		1. If $U \subseteq A$ then we define the \textbf{image} of $U$ as
			\[
				f(U) = \{y \in B: \exists x \in U s.t.\ f(x) = y \} = \{f(x): x\in U\}
			\] 
		
		2. If $V \subseteq B$ then we define the \textbf{pre-image} of $V$ as 
			\[
				f^{-1}(V) = \{x \in A: f(x) \in V\}
			\]
	\end{definition}

	\begin{definition}
		Let $\trans{f}{A}{B}$ be a \underbar{function}, we say that
		\begin{enumerate}
			\item $f$ is \textbf{injective} if and only if
				\[
					f(x) = f(y) \implies x = y,\ \forall x, y \in A
				\]
			\item $f$ is \textbf{surjective} if and only if
				\[
					\forall y \in B, \exists x \in A \  s.t.\ f(x) = y.
				\]
			\item $f$ is \textbf{bijective} if and only if it is both injective and surjective.
		\end{enumerate}
	\end{definition}
	
	
	\subsection{Structures on $\R^n$}
	
	\begin{definition}
		The \textbf{Euclidean inner product}, also know as \textbf{dot product}. Given two vectors $\vec{x} = \stuple{x}{1}{n}$ and $\vec{y} = \stuple{y}{1}{n}$ in $\R^n$ we write
			\[
				\langle \vec{x}, \vec{y} \rangle = \vec{x} \cdot \vec{y} := \sum_{i=1}^n {x_i y_i}
			\]
	\end{definition}
	
	\begin{proposition}
		Let $\vec{x}, \vec{y}, \vec{z} \in \R^n$ and $c \in \R$, then the inner product satisfies
		\begin{enumerate}
			\item \textbf{Symmetry: } $\inner{x}{y} = \inner{y}{x}$.
			\item \textbf{Non-negative: } $\inner{x}{x} \geq 0$ and $\inner{x}{x} = 0 \iff \vec{x} = \vec{0}$.
			\item \textbf{Linearity: } $\Inner{c \vec{x} + \vec{y}}{\vec{z}} = c\inner{x}{z} + \inner{y}{z}$
		\end{enumerate}
	\end{proposition}
	
	
	\begin{theorem}
		(\textbf{Cauchy-Schwarz inequality}) Let $\vec{x}, \vec{y} \in \R^n$ then
		\[
			|\inner{x}{y}| \leq \norm{\vec{x}} \norm{\vec{y}}
		\]
	\end{theorem}
	
	\begin{proposition}
		Let $\vec{x}, \vec{y} \in R^n$ and $c \in \R$, the norm $\norm{\cdot}$ satisfies the following properties:
		\begin{enumerate}
			\item \textbf{Non-degeneracy: } $\norm{\vec{x}} \geq 0$ and $\norm{\vec{x}} = 0 \iff \vec{x} = \vec{0}$.
			\item \textbf{Normality: } $\norm{c \vec{x}} = |c| \norm{\vec{x}}$.
			\item \textbf{Triangle Inequality: } $\norm{\vec{x}} + \norm{\vec{y}} \geq \norm{\vec{x} + \vec{y}}$
		\end{enumerate}
	\end{proposition}
	
	
	
	
	
	
	
	
	
	\textcolor{red}{
	\underbar{STOP: PAGE14}}
	
	
	
	
	
	
	
	
	
	
	
	
	
	
	
	
	
	
	
	
	
	
	
	
	
	
	
	
	
	
	
	
	
		
	
	
	
	
	
	
	
	
	
	
		
	
	
	
	
	
	
	
	
	
	
	
	
\end{document}