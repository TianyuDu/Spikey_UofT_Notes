\documentclass[]{article}

\usepackage{amsmath}
\usepackage{amssymb}
\usepackage{soul}
\usepackage{spikey}
\usepackage{float}
\usepackage{xcolor}
\usepackage{graphicx}
\usepackage{centernot}

\title{MAT237: Lecture Notes \\ \small Advanced Calculus}
\date{\today}
\author{Tianyu Du}


\newcommand{\ball}[2]{\mathcal{B}({#1}, {#2})}
\newcommand{\ballset}[2]{\{\vec{x} \in \R^n: \Norm{\vec{x} - \vec{#2}} < {#1}\}}

\begin{document}
    \maketitle
    \tableofcontents
    
    \newpage
    
    \section{Lecture 1 September 6 2018}
        \subsection{The Geometry of Euclidean Space}
        \begin{example}
            Consider $(1, 2) \in \R^2$ as a \ul{point} or a \ul{vector}.
        \end{example}
        
        \begin{remark}
            All vectors in this course are considered as \ul{column vectors}. Reasoning: suppose a linear function $\trans{f}{\R^n}{\R^m}$, then the transformation can be implemented as 
            \[
                f(\vec{x}) = \tbf{A}\vec{x},\ \tbf{A} \in \rmspace{m}{n}
            \]
            if $\vec{x}$ is a column vector.
        \end{remark}
        
        \begin{definition}
            Let $\vec{a}, \vec{b} \in \R^n$, the \textbf{dot product} $\trans{\cdot}{\R^n \times \R^n}{\R}$ is defined as,
            \[
                \dotprod{a}{b} = \sum_{i}{a_i b_i}
            \]
        \end{definition}
        
        \begin{definition}
            Let $\vec{a} \in \R^n$, the \textbf{Euclidean norm} $\trans{\Norm{\cdot}}{\R^n}{\R}$ is defined as 
            \[
                \norm{a} = \sqrt{\dotprod{a}{b}}
            \]
        \end{definition}
        
        \paragraph{Interpretation} the Euclidean norm of $\vec{a}$, $\norm{a}$ is the \ul{length} of $\vec{a}$, or the \ul{distance} of $\vec{a}$ from the origin. And $\Norm{\vec{a} - \vec{b}}$ is the distance from $\vec{a}$ to $\vec{b}$.
        
        \begin{definition}
            Two vectors $\vec{a}, \vec{b} \in \R^n$ is \textbf{orthogonal} if and only if
            \[
                \dotprod{a}{b} = 0
            \]
        \end{definition}
        
        \begin{theorem}
            \textbf{(Cauchy Schwarz inequality)}
            \[
                | \dotprod{a}{b} | \leq \norm{a} \norm{b}
            \]
        \end{theorem}
        
        \begin{theorem}
        \textbf{(Triangle inequality)}
            \[
                \Norm{\vec{a} + \vec{b}} \leq \norm{a} + \norm{b}
            \]
        \end{theorem}
        
        \begin{theorem}
            \[
                \dotprod{a}{b} = \norm{a} \norm{b} \cos{\theta}
            \]
            where $\theta$ is the angle between $\vec{a}$ and $\vec{b}$
        \end{theorem}
        
        \begin{definition}
            If $\vec{u} \in \Rn$ is a \textbf{unit vector} if  
            \[
                \norm{u} = 1
            \]
        \end{definition}
        
        \begin{definition}
            The \textbf{projection} of $\vec{a}$ onto the line through $\vec{u}$ is defined as 
            \[
                (\dotprod{u}{a})\vec{u}
            \]
        \end{definition}
        
        \subsection{Subspaces of $\R^n$}
        \begin{definition}
            A subspace $V$ if $\R^n$ is a \ul{subset} of $\R^n$ such that
            \[
                \vec{a}, \vec{b} \in V \land
                c_1, c_2 \in \R
                \implies c_1 \vec{a} + c_2 \vec{b} \in V
            \]
        \end{definition}
        
        \begin{example}
            Suppose
            \[
                \tbf{A} = \begin{pmatrix}
                    1 & 3 \\
                    2 & 7 \\
                    -1 & 0 \\
                \end{pmatrix}
            \]
            And consider 
            \[
                V = \{\tbf{A} \vec{x}: \vec{x} \in \R^n\}
            \]
            $V$ is a subspace with dimension 2.
        \end{example}
        
        \begin{theorem}
            Let $\tbf{A} \in \rmspace{m}{n}$ with $m > n$ and columns are independent then $V = \{\tbf{A}\vec{x}: \vec{x} \in \R^n\}$ is a n-dimensional subspace of $\R^n$.
        \end{theorem}
        
        \begin{example} 
            Consider 
            \[
                \tbf{A} = \begin{pmatrix}
                    3 & 1 & 0 \\ 1 & 9 & -2
                \end{pmatrix}
            \] and 
            \[
                V = \{\vec{x} \in \R^3: \tbf{A} \vec{x} = \vec{0}\}
            \]
            Then $V$ is a 1-dimensional subspace of $\R^3$.
        \end{example}
        
        \begin{theorem}
            $\tbf{A} \in \rmspace{m}{n}$ and $m < n$ and rows are linearly independent, then $\{ \vec{x} \in \R^n: \tbf{A}\vec{x} = \vec{0} \}$ is a $(n-m)$ dimensional subspace.
        \end{theorem}
        
        \subsection{Cross Product}
            (\emph{Only available in $\R^3$}) is a way to multiplying two vectors in $\R^3$ to get another vector in $\R^3$.
            \begin{definition}
                Let $\vec{a}, \vec{b} \in \R^3$ then the \tbf{cross product} $\trans{\times}{\R^6}{\R^3}$ is defined as
                \begin{gather*}
                    \vec{a} \times \vec{b} := det(
                        \begin{bmatrix}
                            \vec{i} & \vec{j} & \vec{k} \\
                            a_1 & a_2 & a_3 \\
                            b_1 & b_2 & b_3 \\
                        \end{bmatrix}
                    ) \\
                    \tx{ where } \vec{i} = (1,0,0),\ \vec{j} = (0,1,0),\ \vec{k} = (0,0,1)
                \end{gather*}
            \end{definition}
            
            \begin{remark} $\vec{a} \times \vec{b}$ is the vector such that
                \begin{enumerate}
                    \item orthogonal to both $\vec{a}$ and $\vec{b}$.
                    \item it's length is $\norm{a}\norm{b}\sin{\theta}$.
                \end{enumerate}
            \end{remark}
            
            \begin{proposition} Let $\vec{a}, \vec{b} \in \R^3$, then
                \begin{enumerate}
                    \item $\vec{a} \times \vec{b} = \vec{b} \times \vec{a}$
                    \item $\vec{a} \times \vec{a} = \vec{0}$
                    \item $(c_1 \vec{a_1} + c_2 \vec{a_2}) \times \vec{b} = c_1 (\vec{a_1} \times \vec{b_1}) + c_2 (\vec{a_2} \times \vec{b_2})$
                    \item $(\vec{a} \times \vec{b}) \times \vec{c} \textcolor{red}{\ \neq\ } \vec{}{a} \times (\vec{b} \times \vec{c})$
                \end{enumerate}
            \end{proposition}
    \subsection{Functions of Several Variables}
        \begin{remark}
            Idea of differential calculus: more general general functions can then be approximated by linear functions.
        \end{remark}
        
        \begin{definition}
            Consider function $\trans{f}{\R^2}{\R}$, the graph of $f$ is 
            \[
                \{(x,y,z): z = f(x,y)\} \subseteq \R^3
            \]
        \end{definition}
        
	\section{Lecture 2 September 11 2018}
		\subsection{Visualize function with two variables}
		\begin{definition}
			Given $\trans{f}{\R^2}{\R}$ define \textbf{graph} of 
			\[
				\mc{G}(f):=\{(x,y,z):z=f(x,y)\}
			\]
			and the \textbf{level set} of $f$ is the set $\{(x,y) : f(x,y) = c\}$, with several values of $c$, it's called \textbf{contour plot}.
		\end{definition}
		\begin{example}
			$f(x,y) = \frac{x^2}{4}-\frac{y^2}{9}$.
		\end{example}
		\begin{definition}
			Consider function $\trans{f}{\R^3}{\R}$ we still define the graph of it as 
			\[
				\mc{G}(f) := \{(x,y,z,w): w = f(x,y,z)\} \subseteq \R^4
			\]
			and the \textbf{level sets} (\emph{level surfaces}) of $f$ are defined as
			\[
				\{(x,y,z): f(x,y,z) = c\} \subseteq \R^3
			\]
		\end{definition}
		
		\begin{definition}
			Consider real value function $\trans{f}{\R^n}{\R}$, it's graph is a subset of $\R^{n+1}$ and the contour is a subset of $\R^n$.
		\end{definition}
		
		\subsection{Subsets of $\R^n$}
		\begin{definition}
			Given $r > 0$ and $\vec{a} \in \R^n$, the \textbf{open ball} of radius $r$ centred at $\vec{a}$ is defined as 
			\[
				\mc{B}(r, \vec{a}) :=
				\{\vec{x} \in \R^n: \Norm{\vec{x} - \vec{a}}\ \textcolor{red}{<}\ r\}
			\]
		\end{definition}
		
		\begin{definition}
			The \textbf{sphere} of radius $r$ centred at $\vec{a}$ is defined as 
			\[
				\{\vec{x} \in \R^n: \Norm{\vec{x} - \vec{a}}\ \textcolor{red}{=}\ r\}
			\]
		\end{definition}
		
		\begin{definition}
			Let $S \subseteq \R^n$, then $S$ is \textbf{bounded} if and only if
			\[
				\exists r > 0\ s.t.\ S \subseteq \mc{B}(r, \vec{0})
			\]
		\end{definition}
		
		\begin{example}
			\begin{gather*}
				S_1 = \{(x,y,z): x^2 + y^2 - \cos{e^{e^z}} \leq 5\} \tx{ Unbounded}\\
				S_2 = \{(x,y,z): x^2 + y^2 + z^2 - \cos{e^{e^z}} \leq 5\} \tx{ Bounded}\\
				S_3 = \{(x,y): xy=-1\} \tx{ Unbounded} \\
			\end{gather*}
		\end{example}
	\section{Lecture 3 September 13 2018}
	
	\begin{definition}
		Let $S \subseteq \R^n$, the \textbf{complement} of $S$ in $\R^n$ denoted as $S^c$ is defined as 
		\[
			S^c:= \{\vec{x} \in \R^n: \vec{x} \notin S\}
		\]
	\end{definition}
	
	\begin{definition}
		A point $\vec{x} \in \R^n$ and let $S \subseteq \R^n$ then $\vec{x}$ is in the \textbf{interior} of $S$, denoted as $\vec{x} \in S^{int}$ if
		
		\[
			\exists \epsilon > 0 \ s.t.\ \ball{\epsilon}{\vec{x}} \subseteq S
		\]
	\end{definition}
	
	\begin{definition}
		$\vec{x}$ is in the \textbf{boundary} of $S$, denoted as $\vec{x} \in \partial S$, if 
		\[
			\forall \epsilon > 0,\ 
			\ball{\epsilon}{\vec{x}} \cap S \neq \emptyset 
			\land 
			\ball{\epsilon}{\vec{x}} \cap S^c \neq \emptyset
		\]
	\end{definition}
	
	\begin{definition}
		$\vec{x}$ is in the \textbf{closure} of $S$, denoted as $\vec{x} \in \overline{S}$
		\[
			\forall \epsilon > 0,\ \ball{\epsilon}{\vec{x}} \cap S \neq \emptyset
		\]
	\end{definition}

	\begin{theorem}
		Notice that 
		\[
			\overline{S} = \partial S \cup S^{int}
		\]
	\end{theorem}
	
	\begin{remark}
		Every point of $S$ is either an interior point or a boundary point.
	\end{remark}
	
	\begin{example}
		\[
			S = \ball{r}{\vec{a}} = \{\vec{x}: \Norm{\vec{x} - \vec{a}} < r\}
		\]
		Claim (true):
		\begin{enumerate}
			\item $S^{int} = S$ 
			\item $\partial S = \{\vec{x}: \Norm{\vec{x} - \vec{a}} = r\}$
			\item $\overline{S} = \{\vec{x}: \Norm{\vec{x} - \vec{a}} \leq r\} $
		\end{enumerate}
	\end{example}
	
	\begin{example}
		Consider 
		\[
			S = \{x \in (0, 1): x \in \mathbb{Q}\} \subseteq \R
		\]
		Claim (true):
		\begin{enumerate}
			\item $S^{int} = \emptyset$
			\item $\partial S = [0,1]$
			\item $\overline{S} = [0,1]$
		\end{enumerate}
	\end{example}
	
	\begin{theorem}
		For all set $S \subseteq \R^n$, 
		\[
			S^{int} \subseteq S \subseteq \overline{S}
		\]
	\end{theorem}
	\begin{proof}
		Let $\vec{x} \in S^{int}$, by definition of interior points, $\exists \epsilon > 0\ s.t.\ \ball{\epsilon}{\vec{x}} \subseteq S$, \\
		Since $\vec{x} \in \ball{\epsilon}{\vec{x}}$ by definition of open ball $\implies \vec{x} \in S\ \forall \vec{x} \in S^{int} \implies S^{int} \subseteq S$\\
		Since $\overline{S} = S^{int} \cup \partial S$, therefore $S^{int} \subseteq \overline{S}$. 
	\end{proof}
	
	\begin{theorem}
		For all $S \subseteq \R^n$, 
		\[
			\partial S = \partial (S^c)
		\]
	\end{theorem}
	
	\begin{proof}
		Let $\vec{x} \in \partial (S^c)$ \\
		$\iff \forall \epsilon > 0,\ \ball{\epsilon}{\vec{x}} \cap S \neq \emptyset \land \ball{\epsilon}{\vec{x}} \cap S^c \neq \emptyset$ \\
		$\iff \forall \epsilon > 0, \ \ball{\epsilon}{\vec{x}} \cap (S^c)^c \neq \emptyset \land \ball{\epsilon}{\vec{x}} \cap S^c \neq \emptyset$ \\
		$\iff \vec{x} \in \partial S$
	\end{proof}
	
	\begin{definition}
		A set $S \subseteq \R^n$ is \textbf{open} if $S = S^{int}$ and is \textbf{closed} if $S = \overline{S}$.
	\end{definition}
	
	\begin{remark}
		A set $S$ can be both open and closed or neither open or closed.
	\end{remark}
	
	\begin{example}
		Consider set $S = \R^n$, $\partial S = \emptyset$ then $S = S^{int} = \overline{S}$ and $S$ is \ul{both} open and closed.
	\end{example}
	
	\begin{example}
		Consider $S = \ball{r}{\vec{a}} \subseteq \R^n$, and $S = S^{int} \implies S$ is open.
	\end{example}
	
	\begin{example}
		Consider $S = \emptyset$, $S = S^{int} = \partial S = \overline{S} = \emptyset$ and $S$ is \ul{both} open and closed.
	\end{example}
	
	\begin{example}
		Consider $S = \mathbb{Q}$, $S^{int} = \emptyset$ and $\partial S = \R$, $S$ is \ul{neither} open or closed.
	\end{example}
	
	\begin{remark}
		Most sets are neither open or closed.
	\end{remark}
	
	\begin{theorem}
		Let $S\subseteq \R^n$ be a set, the following statements are equivalent,
		\begin{enumerate}
			\item $S \subseteq \R^n$ is an open set.
			\item $S \subseteq S^{int}$
			\item $\forall \vec{x} \in S,\ \exists\ \epsilon > 0,\ s.t.\ \ball{\epsilon}{\vec{x}} \subseteq S$
		\end{enumerate}
	\end{theorem}
	
	\begin{theorem}
		Let $T \subseteq \R^n$, the following statements are equivalent,
		\begin{enumerate}
			\item $T$ is a closed set.
			\item $\partial T \subseteq T$
			\item $T^c$ is open.
		\end{enumerate}
	\end{theorem}
	
	\begin{proof}
		Let $T$ be a closed set, by definition, $\partial T \subseteq T$. \\
		By theorem 3.3, $\partial(T^c) \subseteq T$, \\
		$\iff \partial {T^c} \centernot \subseteq T^c$ \\
		$\iff$ no points in $T^c$ is boundary point \\
		$\iff \forall \vec{x} \in T^c,\ \neg(\forall \epsilon > 0,\ \ball{\epsilon}{\vec{x}} \cap T^c \neq \emptyset \land \ball{\epsilon}{\vec{x}} \cap T \neq \emptyset)$ \\
		$\iff \forall \vec{x} \in T^c, \exists \epsilon > 0,\ s.t.\ \ball{\epsilon}{\vec{x}} \cap T^c = \emptyset \lor \ball{\epsilon}{\vec{x}} \cap T = \emptyset$ \\
		Clearly, since $\vec{x} \in T^c$, $\ball{\epsilon}{\vec{x}} \cap T^c \neq \emptyset$ \\
		$\iff \forall \vec{x} \in T^c, \exists \epsilon > 0,\ s.t.\ \ball{\epsilon}{\vec{x}} \cap (T^c)^c = \emptyset$,\\
		By definition of open set, $T^c$ is open.
	\end{proof}
	
\end{document}





























