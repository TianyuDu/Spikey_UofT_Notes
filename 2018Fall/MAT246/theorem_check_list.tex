\documentclass{article}
\author{Tianyu Du}
\date{\today}
\title{MAT246 Concepts in Abstract Mathematics \\ \small Midterm Theorem Checklist}

\usepackage{amsmath}
\usepackage{amssymb}
\usepackage{spikey}
\usepackage{soul}

\begin{document}
	\maketitle
	\section{Chapter 1}
	\begin{lemma}[1.1.1]
		Every natural numbers \ul{greater than 1} has a prime divisor.
		\begin{proof}
			Iteratively factorize $n$.
		\end{proof}
	\end{lemma}
	
	\begin{theorem}[1.1.2]
		There's no largest prime.
		\begin{proof}
			(Contradiction) Let $p_n$ be the largest prime and consider \[M=p_1 p_2 \dots p_n + 1\]
		\end{proof}
	\end{theorem}
	
	\section{Chapter 2}
	\begin{theorem}[2.1.1 PMI] $S \subset \mathbb{N}$ and 
	\begin{enumerate}
		\item $1 \in S$.
		\item $k \in S \implies k+1 \in S,\ \forall k \in \mathbb{N}$.
	\end{enumerate}
	then $S = \mathbb{N}$.
		\begin{proof}
			(WOP and Contradiction) \\
			Let $T = S^c$ and show that $T = \emptyset$.
		\end{proof}
	\end{theorem}
	
	\begin{theorem}[2.1.2 WOP]
		$T \subset \mathbb{N} \land T \neq \emptyset \implies \exists \min\{T\}$
		\begin{proof}
			(Contradiction) \\
			Let $T \neq \emptyset$, $t = \min\{T\}$. \\
			Then $t-1 \notin T \implies t-1 \in S \implies t-1+1 = t \in S$.
		\end{proof}
	\end{theorem}
	
	\begin{theorem}
		
	\end{theorem}
	
\end{document}